
% Default to the notebook output style

    


% Inherit from the specified cell style.




    
\documentclass[11pt]{article}

    
    
    \usepackage[T1]{fontenc}
    % Nicer default font (+ math font) than Computer Modern for most use cases
    \usepackage{mathpazo}

    % Basic figure setup, for now with no caption control since it's done
    % automatically by Pandoc (which extracts ![](path) syntax from Markdown).
    \usepackage{graphicx}
    % We will generate all images so they have a width \maxwidth. This means
    % that they will get their normal width if they fit onto the page, but
    % are scaled down if they would overflow the margins.
    \makeatletter
    \def\maxwidth{\ifdim\Gin@nat@width>\linewidth\linewidth
    \else\Gin@nat@width\fi}
    \makeatother
    \let\Oldincludegraphics\includegraphics
    % Set max figure width to be 80% of text width, for now hardcoded.
    \renewcommand{\includegraphics}[1]{\Oldincludegraphics[width=.8\maxwidth]{#1}}
    % Ensure that by default, figures have no caption (until we provide a
    % proper Figure object with a Caption API and a way to capture that
    % in the conversion process - todo).
    \usepackage{caption}
    \DeclareCaptionLabelFormat{nolabel}{}
    \captionsetup{labelformat=nolabel}

    \usepackage{adjustbox} % Used to constrain images to a maximum size 
    \usepackage{xcolor} % Allow colors to be defined
    \usepackage{enumerate} % Needed for markdown enumerations to work
    \usepackage{geometry} % Used to adjust the document margins
    \usepackage{amsmath} % Equations
    \usepackage{amssymb} % Equations
    \usepackage{textcomp} % defines textquotesingle
    % Hack from http://tex.stackexchange.com/a/47451/13684:
    \AtBeginDocument{%
        \def\PYZsq{\textquotesingle}% Upright quotes in Pygmentized code
    }
    \usepackage{upquote} % Upright quotes for verbatim code
    \usepackage{eurosym} % defines \euro
    \usepackage[mathletters]{ucs} % Extended unicode (utf-8) support
    \usepackage[utf8x]{inputenc} % Allow utf-8 characters in the tex document
    \usepackage{fancyvrb} % verbatim replacement that allows latex
    \usepackage{grffile} % extends the file name processing of package graphics 
                         % to support a larger range 
    % The hyperref package gives us a pdf with properly built
    % internal navigation ('pdf bookmarks' for the table of contents,
    % internal cross-reference links, web links for URLs, etc.)
    \usepackage{hyperref}
    \usepackage{longtable} % longtable support required by pandoc >1.10
    \usepackage{booktabs}  % table support for pandoc > 1.12.2
    \usepackage[inline]{enumitem} % IRkernel/repr support (it uses the enumerate* environment)
    \usepackage[normalem]{ulem} % ulem is needed to support strikethroughs (\sout)
                                % normalem makes italics be italics, not underlines
    

    
    
    % Colors for the hyperref package
    \definecolor{urlcolor}{rgb}{0,.145,.698}
    \definecolor{linkcolor}{rgb}{.71,0.21,0.01}
    \definecolor{citecolor}{rgb}{.12,.54,.11}

    % ANSI colors
    \definecolor{ansi-black}{HTML}{3E424D}
    \definecolor{ansi-black-intense}{HTML}{282C36}
    \definecolor{ansi-red}{HTML}{E75C58}
    \definecolor{ansi-red-intense}{HTML}{B22B31}
    \definecolor{ansi-green}{HTML}{00A250}
    \definecolor{ansi-green-intense}{HTML}{007427}
    \definecolor{ansi-yellow}{HTML}{DDB62B}
    \definecolor{ansi-yellow-intense}{HTML}{B27D12}
    \definecolor{ansi-blue}{HTML}{208FFB}
    \definecolor{ansi-blue-intense}{HTML}{0065CA}
    \definecolor{ansi-magenta}{HTML}{D160C4}
    \definecolor{ansi-magenta-intense}{HTML}{A03196}
    \definecolor{ansi-cyan}{HTML}{60C6C8}
    \definecolor{ansi-cyan-intense}{HTML}{258F8F}
    \definecolor{ansi-white}{HTML}{C5C1B4}
    \definecolor{ansi-white-intense}{HTML}{A1A6B2}

    % commands and environments needed by pandoc snippets
    % extracted from the output of `pandoc -s`
    \providecommand{\tightlist}{%
      \setlength{\itemsep}{0pt}\setlength{\parskip}{0pt}}
    \DefineVerbatimEnvironment{Highlighting}{Verbatim}{commandchars=\\\{\}}
    % Add ',fontsize=\small' for more characters per line
    \newenvironment{Shaded}{}{}
    \newcommand{\KeywordTok}[1]{\textcolor[rgb]{0.00,0.44,0.13}{\textbf{{#1}}}}
    \newcommand{\DataTypeTok}[1]{\textcolor[rgb]{0.56,0.13,0.00}{{#1}}}
    \newcommand{\DecValTok}[1]{\textcolor[rgb]{0.25,0.63,0.44}{{#1}}}
    \newcommand{\BaseNTok}[1]{\textcolor[rgb]{0.25,0.63,0.44}{{#1}}}
    \newcommand{\FloatTok}[1]{\textcolor[rgb]{0.25,0.63,0.44}{{#1}}}
    \newcommand{\CharTok}[1]{\textcolor[rgb]{0.25,0.44,0.63}{{#1}}}
    \newcommand{\StringTok}[1]{\textcolor[rgb]{0.25,0.44,0.63}{{#1}}}
    \newcommand{\CommentTok}[1]{\textcolor[rgb]{0.38,0.63,0.69}{\textit{{#1}}}}
    \newcommand{\OtherTok}[1]{\textcolor[rgb]{0.00,0.44,0.13}{{#1}}}
    \newcommand{\AlertTok}[1]{\textcolor[rgb]{1.00,0.00,0.00}{\textbf{{#1}}}}
    \newcommand{\FunctionTok}[1]{\textcolor[rgb]{0.02,0.16,0.49}{{#1}}}
    \newcommand{\RegionMarkerTok}[1]{{#1}}
    \newcommand{\ErrorTok}[1]{\textcolor[rgb]{1.00,0.00,0.00}{\textbf{{#1}}}}
    \newcommand{\NormalTok}[1]{{#1}}
    
    % Additional commands for more recent versions of Pandoc
    \newcommand{\ConstantTok}[1]{\textcolor[rgb]{0.53,0.00,0.00}{{#1}}}
    \newcommand{\SpecialCharTok}[1]{\textcolor[rgb]{0.25,0.44,0.63}{{#1}}}
    \newcommand{\VerbatimStringTok}[1]{\textcolor[rgb]{0.25,0.44,0.63}{{#1}}}
    \newcommand{\SpecialStringTok}[1]{\textcolor[rgb]{0.73,0.40,0.53}{{#1}}}
    \newcommand{\ImportTok}[1]{{#1}}
    \newcommand{\DocumentationTok}[1]{\textcolor[rgb]{0.73,0.13,0.13}{\textit{{#1}}}}
    \newcommand{\AnnotationTok}[1]{\textcolor[rgb]{0.38,0.63,0.69}{\textbf{\textit{{#1}}}}}
    \newcommand{\CommentVarTok}[1]{\textcolor[rgb]{0.38,0.63,0.69}{\textbf{\textit{{#1}}}}}
    \newcommand{\VariableTok}[1]{\textcolor[rgb]{0.10,0.09,0.49}{{#1}}}
    \newcommand{\ControlFlowTok}[1]{\textcolor[rgb]{0.00,0.44,0.13}{\textbf{{#1}}}}
    \newcommand{\OperatorTok}[1]{\textcolor[rgb]{0.40,0.40,0.40}{{#1}}}
    \newcommand{\BuiltInTok}[1]{{#1}}
    \newcommand{\ExtensionTok}[1]{{#1}}
    \newcommand{\PreprocessorTok}[1]{\textcolor[rgb]{0.74,0.48,0.00}{{#1}}}
    \newcommand{\AttributeTok}[1]{\textcolor[rgb]{0.49,0.56,0.16}{{#1}}}
    \newcommand{\InformationTok}[1]{\textcolor[rgb]{0.38,0.63,0.69}{\textbf{\textit{{#1}}}}}
    \newcommand{\WarningTok}[1]{\textcolor[rgb]{0.38,0.63,0.69}{\textbf{\textit{{#1}}}}}
    
    
    % Define a nice break command that doesn't care if a line doesn't already
    % exist.
    \def\br{\hspace*{\fill} \\* }
    % Math Jax compatability definitions
    \def\gt{>}
    \def\lt{<}
    % Document parameters
    \title{Analyze\_ab\_test\_results\_notebook}
    
    
    

    % Pygments definitions
    
\makeatletter
\def\PY@reset{\let\PY@it=\relax \let\PY@bf=\relax%
    \let\PY@ul=\relax \let\PY@tc=\relax%
    \let\PY@bc=\relax \let\PY@ff=\relax}
\def\PY@tok#1{\csname PY@tok@#1\endcsname}
\def\PY@toks#1+{\ifx\relax#1\empty\else%
    \PY@tok{#1}\expandafter\PY@toks\fi}
\def\PY@do#1{\PY@bc{\PY@tc{\PY@ul{%
    \PY@it{\PY@bf{\PY@ff{#1}}}}}}}
\def\PY#1#2{\PY@reset\PY@toks#1+\relax+\PY@do{#2}}

\expandafter\def\csname PY@tok@w\endcsname{\def\PY@tc##1{\textcolor[rgb]{0.73,0.73,0.73}{##1}}}
\expandafter\def\csname PY@tok@c\endcsname{\let\PY@it=\textit\def\PY@tc##1{\textcolor[rgb]{0.25,0.50,0.50}{##1}}}
\expandafter\def\csname PY@tok@cp\endcsname{\def\PY@tc##1{\textcolor[rgb]{0.74,0.48,0.00}{##1}}}
\expandafter\def\csname PY@tok@k\endcsname{\let\PY@bf=\textbf\def\PY@tc##1{\textcolor[rgb]{0.00,0.50,0.00}{##1}}}
\expandafter\def\csname PY@tok@kp\endcsname{\def\PY@tc##1{\textcolor[rgb]{0.00,0.50,0.00}{##1}}}
\expandafter\def\csname PY@tok@kt\endcsname{\def\PY@tc##1{\textcolor[rgb]{0.69,0.00,0.25}{##1}}}
\expandafter\def\csname PY@tok@o\endcsname{\def\PY@tc##1{\textcolor[rgb]{0.40,0.40,0.40}{##1}}}
\expandafter\def\csname PY@tok@ow\endcsname{\let\PY@bf=\textbf\def\PY@tc##1{\textcolor[rgb]{0.67,0.13,1.00}{##1}}}
\expandafter\def\csname PY@tok@nb\endcsname{\def\PY@tc##1{\textcolor[rgb]{0.00,0.50,0.00}{##1}}}
\expandafter\def\csname PY@tok@nf\endcsname{\def\PY@tc##1{\textcolor[rgb]{0.00,0.00,1.00}{##1}}}
\expandafter\def\csname PY@tok@nc\endcsname{\let\PY@bf=\textbf\def\PY@tc##1{\textcolor[rgb]{0.00,0.00,1.00}{##1}}}
\expandafter\def\csname PY@tok@nn\endcsname{\let\PY@bf=\textbf\def\PY@tc##1{\textcolor[rgb]{0.00,0.00,1.00}{##1}}}
\expandafter\def\csname PY@tok@ne\endcsname{\let\PY@bf=\textbf\def\PY@tc##1{\textcolor[rgb]{0.82,0.25,0.23}{##1}}}
\expandafter\def\csname PY@tok@nv\endcsname{\def\PY@tc##1{\textcolor[rgb]{0.10,0.09,0.49}{##1}}}
\expandafter\def\csname PY@tok@no\endcsname{\def\PY@tc##1{\textcolor[rgb]{0.53,0.00,0.00}{##1}}}
\expandafter\def\csname PY@tok@nl\endcsname{\def\PY@tc##1{\textcolor[rgb]{0.63,0.63,0.00}{##1}}}
\expandafter\def\csname PY@tok@ni\endcsname{\let\PY@bf=\textbf\def\PY@tc##1{\textcolor[rgb]{0.60,0.60,0.60}{##1}}}
\expandafter\def\csname PY@tok@na\endcsname{\def\PY@tc##1{\textcolor[rgb]{0.49,0.56,0.16}{##1}}}
\expandafter\def\csname PY@tok@nt\endcsname{\let\PY@bf=\textbf\def\PY@tc##1{\textcolor[rgb]{0.00,0.50,0.00}{##1}}}
\expandafter\def\csname PY@tok@nd\endcsname{\def\PY@tc##1{\textcolor[rgb]{0.67,0.13,1.00}{##1}}}
\expandafter\def\csname PY@tok@s\endcsname{\def\PY@tc##1{\textcolor[rgb]{0.73,0.13,0.13}{##1}}}
\expandafter\def\csname PY@tok@sd\endcsname{\let\PY@it=\textit\def\PY@tc##1{\textcolor[rgb]{0.73,0.13,0.13}{##1}}}
\expandafter\def\csname PY@tok@si\endcsname{\let\PY@bf=\textbf\def\PY@tc##1{\textcolor[rgb]{0.73,0.40,0.53}{##1}}}
\expandafter\def\csname PY@tok@se\endcsname{\let\PY@bf=\textbf\def\PY@tc##1{\textcolor[rgb]{0.73,0.40,0.13}{##1}}}
\expandafter\def\csname PY@tok@sr\endcsname{\def\PY@tc##1{\textcolor[rgb]{0.73,0.40,0.53}{##1}}}
\expandafter\def\csname PY@tok@ss\endcsname{\def\PY@tc##1{\textcolor[rgb]{0.10,0.09,0.49}{##1}}}
\expandafter\def\csname PY@tok@sx\endcsname{\def\PY@tc##1{\textcolor[rgb]{0.00,0.50,0.00}{##1}}}
\expandafter\def\csname PY@tok@m\endcsname{\def\PY@tc##1{\textcolor[rgb]{0.40,0.40,0.40}{##1}}}
\expandafter\def\csname PY@tok@gh\endcsname{\let\PY@bf=\textbf\def\PY@tc##1{\textcolor[rgb]{0.00,0.00,0.50}{##1}}}
\expandafter\def\csname PY@tok@gu\endcsname{\let\PY@bf=\textbf\def\PY@tc##1{\textcolor[rgb]{0.50,0.00,0.50}{##1}}}
\expandafter\def\csname PY@tok@gd\endcsname{\def\PY@tc##1{\textcolor[rgb]{0.63,0.00,0.00}{##1}}}
\expandafter\def\csname PY@tok@gi\endcsname{\def\PY@tc##1{\textcolor[rgb]{0.00,0.63,0.00}{##1}}}
\expandafter\def\csname PY@tok@gr\endcsname{\def\PY@tc##1{\textcolor[rgb]{1.00,0.00,0.00}{##1}}}
\expandafter\def\csname PY@tok@ge\endcsname{\let\PY@it=\textit}
\expandafter\def\csname PY@tok@gs\endcsname{\let\PY@bf=\textbf}
\expandafter\def\csname PY@tok@gp\endcsname{\let\PY@bf=\textbf\def\PY@tc##1{\textcolor[rgb]{0.00,0.00,0.50}{##1}}}
\expandafter\def\csname PY@tok@go\endcsname{\def\PY@tc##1{\textcolor[rgb]{0.53,0.53,0.53}{##1}}}
\expandafter\def\csname PY@tok@gt\endcsname{\def\PY@tc##1{\textcolor[rgb]{0.00,0.27,0.87}{##1}}}
\expandafter\def\csname PY@tok@err\endcsname{\def\PY@bc##1{\setlength{\fboxsep}{0pt}\fcolorbox[rgb]{1.00,0.00,0.00}{1,1,1}{\strut ##1}}}
\expandafter\def\csname PY@tok@kc\endcsname{\let\PY@bf=\textbf\def\PY@tc##1{\textcolor[rgb]{0.00,0.50,0.00}{##1}}}
\expandafter\def\csname PY@tok@kd\endcsname{\let\PY@bf=\textbf\def\PY@tc##1{\textcolor[rgb]{0.00,0.50,0.00}{##1}}}
\expandafter\def\csname PY@tok@kn\endcsname{\let\PY@bf=\textbf\def\PY@tc##1{\textcolor[rgb]{0.00,0.50,0.00}{##1}}}
\expandafter\def\csname PY@tok@kr\endcsname{\let\PY@bf=\textbf\def\PY@tc##1{\textcolor[rgb]{0.00,0.50,0.00}{##1}}}
\expandafter\def\csname PY@tok@bp\endcsname{\def\PY@tc##1{\textcolor[rgb]{0.00,0.50,0.00}{##1}}}
\expandafter\def\csname PY@tok@fm\endcsname{\def\PY@tc##1{\textcolor[rgb]{0.00,0.00,1.00}{##1}}}
\expandafter\def\csname PY@tok@vc\endcsname{\def\PY@tc##1{\textcolor[rgb]{0.10,0.09,0.49}{##1}}}
\expandafter\def\csname PY@tok@vg\endcsname{\def\PY@tc##1{\textcolor[rgb]{0.10,0.09,0.49}{##1}}}
\expandafter\def\csname PY@tok@vi\endcsname{\def\PY@tc##1{\textcolor[rgb]{0.10,0.09,0.49}{##1}}}
\expandafter\def\csname PY@tok@vm\endcsname{\def\PY@tc##1{\textcolor[rgb]{0.10,0.09,0.49}{##1}}}
\expandafter\def\csname PY@tok@sa\endcsname{\def\PY@tc##1{\textcolor[rgb]{0.73,0.13,0.13}{##1}}}
\expandafter\def\csname PY@tok@sb\endcsname{\def\PY@tc##1{\textcolor[rgb]{0.73,0.13,0.13}{##1}}}
\expandafter\def\csname PY@tok@sc\endcsname{\def\PY@tc##1{\textcolor[rgb]{0.73,0.13,0.13}{##1}}}
\expandafter\def\csname PY@tok@dl\endcsname{\def\PY@tc##1{\textcolor[rgb]{0.73,0.13,0.13}{##1}}}
\expandafter\def\csname PY@tok@s2\endcsname{\def\PY@tc##1{\textcolor[rgb]{0.73,0.13,0.13}{##1}}}
\expandafter\def\csname PY@tok@sh\endcsname{\def\PY@tc##1{\textcolor[rgb]{0.73,0.13,0.13}{##1}}}
\expandafter\def\csname PY@tok@s1\endcsname{\def\PY@tc##1{\textcolor[rgb]{0.73,0.13,0.13}{##1}}}
\expandafter\def\csname PY@tok@mb\endcsname{\def\PY@tc##1{\textcolor[rgb]{0.40,0.40,0.40}{##1}}}
\expandafter\def\csname PY@tok@mf\endcsname{\def\PY@tc##1{\textcolor[rgb]{0.40,0.40,0.40}{##1}}}
\expandafter\def\csname PY@tok@mh\endcsname{\def\PY@tc##1{\textcolor[rgb]{0.40,0.40,0.40}{##1}}}
\expandafter\def\csname PY@tok@mi\endcsname{\def\PY@tc##1{\textcolor[rgb]{0.40,0.40,0.40}{##1}}}
\expandafter\def\csname PY@tok@il\endcsname{\def\PY@tc##1{\textcolor[rgb]{0.40,0.40,0.40}{##1}}}
\expandafter\def\csname PY@tok@mo\endcsname{\def\PY@tc##1{\textcolor[rgb]{0.40,0.40,0.40}{##1}}}
\expandafter\def\csname PY@tok@ch\endcsname{\let\PY@it=\textit\def\PY@tc##1{\textcolor[rgb]{0.25,0.50,0.50}{##1}}}
\expandafter\def\csname PY@tok@cm\endcsname{\let\PY@it=\textit\def\PY@tc##1{\textcolor[rgb]{0.25,0.50,0.50}{##1}}}
\expandafter\def\csname PY@tok@cpf\endcsname{\let\PY@it=\textit\def\PY@tc##1{\textcolor[rgb]{0.25,0.50,0.50}{##1}}}
\expandafter\def\csname PY@tok@c1\endcsname{\let\PY@it=\textit\def\PY@tc##1{\textcolor[rgb]{0.25,0.50,0.50}{##1}}}
\expandafter\def\csname PY@tok@cs\endcsname{\let\PY@it=\textit\def\PY@tc##1{\textcolor[rgb]{0.25,0.50,0.50}{##1}}}

\def\PYZbs{\char`\\}
\def\PYZus{\char`\_}
\def\PYZob{\char`\{}
\def\PYZcb{\char`\}}
\def\PYZca{\char`\^}
\def\PYZam{\char`\&}
\def\PYZlt{\char`\<}
\def\PYZgt{\char`\>}
\def\PYZsh{\char`\#}
\def\PYZpc{\char`\%}
\def\PYZdl{\char`\$}
\def\PYZhy{\char`\-}
\def\PYZsq{\char`\'}
\def\PYZdq{\char`\"}
\def\PYZti{\char`\~}
% for compatibility with earlier versions
\def\PYZat{@}
\def\PYZlb{[}
\def\PYZrb{]}
\makeatother


    % Exact colors from NB
    \definecolor{incolor}{rgb}{0.0, 0.0, 0.5}
    \definecolor{outcolor}{rgb}{0.545, 0.0, 0.0}



    
    % Prevent overflowing lines due to hard-to-break entities
    \sloppy 
    % Setup hyperref package
    \hypersetup{
      breaklinks=true,  % so long urls are correctly broken across lines
      colorlinks=true,
      urlcolor=urlcolor,
      linkcolor=linkcolor,
      citecolor=citecolor,
      }
    % Slightly bigger margins than the latex defaults
    
    \geometry{verbose,tmargin=1in,bmargin=1in,lmargin=1in,rmargin=1in}
    
    

    \begin{document}
    
    
    \maketitle
    
    

    
    \hypertarget{analyze-ab-test-results}{%
\subsection{Analyze A/B Test Results}\label{analyze-ab-test-results}}

This project will assure you have mastered the subjects covered in the
statistics lessons. The hope is to have this project be as comprehensive
of these topics as possible. Good luck!

\hypertarget{table-of-contents}{%
\subsection{Table of Contents}\label{table-of-contents}}

\begin{itemize}
\tightlist
\item
  Section \ref{intro}
\item
  Section \ref{probability}
\item
  Section \ref{ab_test}
\item
  Section \ref{regression}
\end{itemize}

 \#\#\# Introduction

A/B tests are very commonly performed by data analysts and data
scientists. It is important that you get some practice working with the
difficulties of these

For this project, you will be working to understand the results of an
A/B test run by an e-commerce website. Your goal is to work through this
notebook to help the company understand if they should implement the new
page, keep the old page, or perhaps run the experiment longer to make
their decision.

\textbf{As you work through this notebook, follow along in the classroom
and answer the corresponding quiz questions associated with each
question.} The labels for each classroom concept are provided for each
question. This will assure you are on the right track as you work
through the project, and you can feel more confident in your final
submission meeting the criteria. As a final check, assure you meet all
the criteria on the
\href{https://review.udacity.com/\#!/projects/37e27304-ad47-4eb0-a1ab-8c12f60e43d0/rubric}{RUBRIC}.

 \#\#\#\# Part I - Probability

To get started, let's import our libraries.

    \begin{Verbatim}[commandchars=\\\{\}]
{\color{incolor}In [{\color{incolor}1}]:} \PY{k+kn}{import} \PY{n+nn}{pandas} \PY{k}{as} \PY{n+nn}{pd}
        \PY{k+kn}{import} \PY{n+nn}{numpy} \PY{k}{as} \PY{n+nn}{np}
        \PY{k+kn}{import} \PY{n+nn}{random}
        \PY{k+kn}{import} \PY{n+nn}{matplotlib}\PY{n+nn}{.}\PY{n+nn}{pyplot} \PY{k}{as} \PY{n+nn}{plt}
        \PY{o}{\PYZpc{}}\PY{k}{matplotlib} inline
        \PY{c+c1}{\PYZsh{}We are setting the seed to assure you get the same answers on quizzes as we set up}
        \PY{n}{random}\PY{o}{.}\PY{n}{seed}\PY{p}{(}\PY{l+m+mi}{42}\PY{p}{)}
\end{Verbatim}


    \texttt{1.} Now, read in the \texttt{ab\_data.csv} data. Store it in
\texttt{df}. \textbf{Use your dataframe to answer the questions in Quiz
1 of the classroom.}

\begin{enumerate}
\def\labelenumi{\alph{enumi}.}
\tightlist
\item
  Read in the dataset and take a look at the top few rows here:
\end{enumerate}

    \begin{Verbatim}[commandchars=\\\{\}]
{\color{incolor}In [{\color{incolor}2}]:} \PY{n}{df} \PY{o}{=} \PY{n}{pd}\PY{o}{.}\PY{n}{read\PYZus{}csv}\PY{p}{(}\PY{l+s+s1}{\PYZsq{}}\PY{l+s+s1}{ab\PYZus{}data.csv}\PY{l+s+s1}{\PYZsq{}}\PY{p}{)}
        \PY{n}{df}\PY{o}{.}\PY{n}{head}\PY{p}{(}\PY{p}{)}
\end{Verbatim}


\begin{Verbatim}[commandchars=\\\{\}]
{\color{outcolor}Out[{\color{outcolor}2}]:}    user\_id                   timestamp      group landing\_page  converted
        0   851104  2017-01-21 22:11:48.556739    control     old\_page          0
        1   804228  2017-01-12 08:01:45.159739    control     old\_page          0
        2   661590  2017-01-11 16:55:06.154213  treatment     new\_page          0
        3   853541  2017-01-08 18:28:03.143765  treatment     new\_page          0
        4   864975  2017-01-21 01:52:26.210827    control     old\_page          1
\end{Verbatim}
            
    \begin{enumerate}
\def\labelenumi{\alph{enumi}.}
\setcounter{enumi}{1}
\tightlist
\item
  Use the below cell to find the number of rows in the dataset.
\end{enumerate}

    \begin{Verbatim}[commandchars=\\\{\}]
{\color{incolor}In [{\color{incolor}3}]:} \PY{c+c1}{\PYZsh{} info() gives us the total count of rows in each column of our dataset. Its a quick check on the number of rows, with also}
        \PY{c+c1}{\PYZsh{} allowing us to see any columns are missing values.}
        \PY{n}{df}\PY{o}{.}\PY{n}{info}\PY{p}{(}\PY{p}{)}
\end{Verbatim}


    \begin{Verbatim}[commandchars=\\\{\}]
<class 'pandas.core.frame.DataFrame'>
RangeIndex: 294478 entries, 0 to 294477
Data columns (total 5 columns):
user\_id         294478 non-null int64
timestamp       294478 non-null object
group           294478 non-null object
landing\_page    294478 non-null object
converted       294478 non-null int64
dtypes: int64(2), object(3)
memory usage: 11.2+ MB

    \end{Verbatim}

    \begin{enumerate}
\def\labelenumi{\alph{enumi}.}
\setcounter{enumi}{2}
\tightlist
\item
  The number of unique users in the dataset.
\end{enumerate}

    \begin{Verbatim}[commandchars=\\\{\}]
{\color{incolor}In [{\color{incolor}4}]:} \PY{c+c1}{\PYZsh{} the nunique function gives us a count of all unique values in a field. We can see from here that of the 294478 rows we have ...}
        \PY{n}{df}\PY{p}{[}\PY{l+s+s1}{\PYZsq{}}\PY{l+s+s1}{user\PYZus{}id}\PY{l+s+s1}{\PYZsq{}}\PY{p}{]}\PY{o}{.}\PY{n}{nunique}\PY{p}{(}\PY{p}{)}
\end{Verbatim}


\begin{Verbatim}[commandchars=\\\{\}]
{\color{outcolor}Out[{\color{outcolor}4}]:} 290584
\end{Verbatim}
            
    \begin{enumerate}
\def\labelenumi{\alph{enumi}.}
\setcounter{enumi}{3}
\tightlist
\item
  The proportion of users converted.
\end{enumerate}

    \begin{Verbatim}[commandchars=\\\{\}]
{\color{incolor}In [{\color{incolor}5}]:} \PY{c+c1}{\PYZsh{} Find the number of users converted. We can do this by takingd the sum of those the converted column}
        \PY{c+c1}{\PYZsh{} because its listed as a 0 \PYZhy{} False and 1 \PYZhy{} True allowing us to add all the 1\PYZsq{}s together.}
        \PY{c+c1}{\PYZsh{} for here we can divide by the total count of rows in the coverted column.}
        \PY{n}{df}\PY{o}{.}\PY{n}{converted}\PY{o}{.}\PY{n}{sum}\PY{p}{(}\PY{p}{)} \PY{o}{/} \PY{n}{df}\PY{o}{.}\PY{n}{converted}\PY{o}{.}\PY{n}{count}\PY{p}{(}\PY{p}{)}
\end{Verbatim}


\begin{Verbatim}[commandchars=\\\{\}]
{\color{outcolor}Out[{\color{outcolor}5}]:} 0.11965919355605512
\end{Verbatim}
            
    \begin{enumerate}
\def\labelenumi{\alph{enumi}.}
\setcounter{enumi}{4}
\tightlist
\item
  The number of times the \texttt{new\_page} and \texttt{treatment}
  don't line up.
\end{enumerate}

    \begin{Verbatim}[commandchars=\\\{\}]
{\color{incolor}In [{\color{incolor}6}]:} \PY{c+c1}{\PYZsh{} Count the number of rows we have were group is treatment and landing page is old\PYZus{}page. This would be when they don\PYZsq{}t line up.}
        \PY{n}{df}\PY{o}{.}\PY{n}{query}\PY{p}{(}\PY{l+s+s1}{\PYZsq{}}\PY{l+s+s1}{group == }\PY{l+s+s1}{\PYZdq{}}\PY{l+s+s1}{treatment}\PY{l+s+s1}{\PYZdq{}}\PY{l+s+s1}{ \PYZam{} landing\PYZus{}page == }\PY{l+s+s1}{\PYZdq{}}\PY{l+s+s1}{old\PYZus{}page}\PY{l+s+s1}{\PYZdq{}}\PY{l+s+s1}{\PYZsq{}}\PY{p}{)}\PY{p}{[}\PY{l+s+s1}{\PYZsq{}}\PY{l+s+s1}{user\PYZus{}id}\PY{l+s+s1}{\PYZsq{}}\PY{p}{]}\PY{o}{.}\PY{n}{count}\PY{p}{(}\PY{p}{)} \PY{o}{+} \PY{n}{df}\PY{o}{.}\PY{n}{query}\PY{p}{(}\PY{l+s+s1}{\PYZsq{}}\PY{l+s+s1}{group == }\PY{l+s+s1}{\PYZdq{}}\PY{l+s+s1}{control}\PY{l+s+s1}{\PYZdq{}}\PY{l+s+s1}{ \PYZam{} landing\PYZus{}page == }\PY{l+s+s1}{\PYZdq{}}\PY{l+s+s1}{new\PYZus{}page}\PY{l+s+s1}{\PYZdq{}}\PY{l+s+s1}{\PYZsq{}}\PY{p}{)}\PY{p}{[}\PY{l+s+s1}{\PYZsq{}}\PY{l+s+s1}{user\PYZus{}id}\PY{l+s+s1}{\PYZsq{}}\PY{p}{]}\PY{o}{.}\PY{n}{count}\PY{p}{(}\PY{p}{)}
\end{Verbatim}


\begin{Verbatim}[commandchars=\\\{\}]
{\color{outcolor}Out[{\color{outcolor}6}]:} 3893
\end{Verbatim}
            
    \begin{Verbatim}[commandchars=\\\{\}]
{\color{incolor}In [{\color{incolor}7}]:} \PY{c+c1}{\PYZsh{} The number of times that the pages don\PYZsq{}t line up is 3893}
\end{Verbatim}


    \begin{enumerate}
\def\labelenumi{\alph{enumi}.}
\setcounter{enumi}{5}
\tightlist
\item
  Do any of the rows have missing values?
\end{enumerate}

    \begin{Verbatim}[commandchars=\\\{\}]
{\color{incolor}In [{\color{incolor}8}]:} \PY{n}{df}\PY{o}{.}\PY{n}{info}\PY{p}{(}\PY{p}{)}
        \PY{c+c1}{\PYZsh{} No all rows have a value}
\end{Verbatim}


    \begin{Verbatim}[commandchars=\\\{\}]
<class 'pandas.core.frame.DataFrame'>
RangeIndex: 294478 entries, 0 to 294477
Data columns (total 5 columns):
user\_id         294478 non-null int64
timestamp       294478 non-null object
group           294478 non-null object
landing\_page    294478 non-null object
converted       294478 non-null int64
dtypes: int64(2), object(3)
memory usage: 11.2+ MB

    \end{Verbatim}

    \texttt{2.} For the rows where \textbf{treatment} is not aligned with
\textbf{new\_page} or \textbf{control} is not aligned with
\textbf{old\_page}, we cannot be sure if this row truly received the new
or old page. Use \textbf{Quiz 2} in the classroom to provide how we
should handle these rows.

\begin{enumerate}
\def\labelenumi{\alph{enumi}.}
\tightlist
\item
  Now use the answer to the quiz to create a new dataset that meets the
  specifications from the quiz. Store your new dataframe in
  \textbf{df2}.
\end{enumerate}

    \begin{Verbatim}[commandchars=\\\{\}]
{\color{incolor}In [{\color{incolor}9}]:} \PY{n}{df2} \PY{o}{=} \PY{n}{df}\PY{o}{.}\PY{n}{query}\PY{p}{(}\PY{l+s+s1}{\PYZsq{}}\PY{l+s+s1}{group == }\PY{l+s+s1}{\PYZdq{}}\PY{l+s+s1}{treatment}\PY{l+s+s1}{\PYZdq{}}\PY{l+s+s1}{ \PYZam{} landing\PYZus{}page == }\PY{l+s+s1}{\PYZdq{}}\PY{l+s+s1}{new\PYZus{}page}\PY{l+s+s1}{\PYZdq{}}\PY{l+s+s1}{ | group == }\PY{l+s+s1}{\PYZdq{}}\PY{l+s+s1}{control}\PY{l+s+s1}{\PYZdq{}}\PY{l+s+s1}{ \PYZam{} landing\PYZus{}page == }\PY{l+s+s1}{\PYZdq{}}\PY{l+s+s1}{old\PYZus{}page}\PY{l+s+s1}{\PYZdq{}}\PY{l+s+s1}{\PYZsq{}}\PY{p}{)}
        \PY{n}{df2}\PY{o}{.}\PY{n}{head}\PY{p}{(}\PY{p}{)}
\end{Verbatim}


\begin{Verbatim}[commandchars=\\\{\}]
{\color{outcolor}Out[{\color{outcolor}9}]:}    user\_id                   timestamp      group landing\_page  converted
        0   851104  2017-01-21 22:11:48.556739    control     old\_page          0
        1   804228  2017-01-12 08:01:45.159739    control     old\_page          0
        2   661590  2017-01-11 16:55:06.154213  treatment     new\_page          0
        3   853541  2017-01-08 18:28:03.143765  treatment     new\_page          0
        4   864975  2017-01-21 01:52:26.210827    control     old\_page          1
\end{Verbatim}
            
    \begin{Verbatim}[commandchars=\\\{\}]
{\color{incolor}In [{\color{incolor}10}]:} \PY{c+c1}{\PYZsh{} Double Check all of the correct rows were removed \PYZhy{} this should be 0}
         \PY{n}{df2}\PY{p}{[}\PY{p}{(}\PY{p}{(}\PY{n}{df2}\PY{p}{[}\PY{l+s+s1}{\PYZsq{}}\PY{l+s+s1}{group}\PY{l+s+s1}{\PYZsq{}}\PY{p}{]} \PY{o}{==} \PY{l+s+s1}{\PYZsq{}}\PY{l+s+s1}{treatment}\PY{l+s+s1}{\PYZsq{}}\PY{p}{)} \PY{o}{==} \PY{p}{(}\PY{n}{df2}\PY{p}{[}\PY{l+s+s1}{\PYZsq{}}\PY{l+s+s1}{landing\PYZus{}page}\PY{l+s+s1}{\PYZsq{}}\PY{p}{]} \PY{o}{==} \PY{l+s+s1}{\PYZsq{}}\PY{l+s+s1}{new\PYZus{}page}\PY{l+s+s1}{\PYZsq{}}\PY{p}{)}\PY{p}{)} \PY{o}{==} \PY{k+kc}{False}\PY{p}{]}\PY{o}{.}\PY{n}{shape}\PY{p}{[}\PY{l+m+mi}{0}\PY{p}{]}
\end{Verbatim}


\begin{Verbatim}[commandchars=\\\{\}]
{\color{outcolor}Out[{\color{outcolor}10}]:} 0
\end{Verbatim}
            
    \texttt{3.} Use \textbf{df2} and the cells below to answer questions for
\textbf{Quiz3} in the classroom.

    \begin{enumerate}
\def\labelenumi{\alph{enumi}.}
\tightlist
\item
  How many unique \textbf{user\_id}s are in \textbf{df2}?
\end{enumerate}

    \begin{Verbatim}[commandchars=\\\{\}]
{\color{incolor}In [{\color{incolor}11}]:} \PY{n}{df2}\PY{p}{[}\PY{l+s+s1}{\PYZsq{}}\PY{l+s+s1}{user\PYZus{}id}\PY{l+s+s1}{\PYZsq{}}\PY{p}{]}\PY{o}{.}\PY{n}{nunique}\PY{p}{(}\PY{p}{)}
\end{Verbatim}


\begin{Verbatim}[commandchars=\\\{\}]
{\color{outcolor}Out[{\color{outcolor}11}]:} 290584
\end{Verbatim}
            
    \begin{enumerate}
\def\labelenumi{\alph{enumi}.}
\setcounter{enumi}{1}
\tightlist
\item
  There is one \textbf{user\_id} repeated in \textbf{df2}. What is it?
\end{enumerate}

    \begin{Verbatim}[commandchars=\\\{\}]
{\color{incolor}In [{\color{incolor}12}]:} \PY{n}{df2}\PY{p}{[}\PY{n}{df2}\PY{o}{.}\PY{n}{duplicated}\PY{p}{(}\PY{p}{[}\PY{l+s+s1}{\PYZsq{}}\PY{l+s+s1}{user\PYZus{}id}\PY{l+s+s1}{\PYZsq{}}\PY{p}{]}\PY{p}{,} \PY{n}{keep}\PY{o}{=}\PY{k+kc}{False}\PY{p}{)}\PY{p}{]}\PY{p}{[}\PY{l+s+s1}{\PYZsq{}}\PY{l+s+s1}{user\PYZus{}id}\PY{l+s+s1}{\PYZsq{}}\PY{p}{]}
\end{Verbatim}


\begin{Verbatim}[commandchars=\\\{\}]
{\color{outcolor}Out[{\color{outcolor}12}]:} 1899    773192
         2893    773192
         Name: user\_id, dtype: int64
\end{Verbatim}
            
    \begin{enumerate}
\def\labelenumi{\alph{enumi}.}
\setcounter{enumi}{2}
\tightlist
\item
  What is the row information for the repeat \textbf{user\_id}?
\end{enumerate}

    \begin{Verbatim}[commandchars=\\\{\}]
{\color{incolor}In [{\color{incolor}13}]:} \PY{n}{df2}\PY{p}{[}\PY{n}{df2}\PY{o}{.}\PY{n}{duplicated}\PY{p}{(}\PY{p}{[}\PY{l+s+s1}{\PYZsq{}}\PY{l+s+s1}{user\PYZus{}id}\PY{l+s+s1}{\PYZsq{}}\PY{p}{]}\PY{p}{,} \PY{n}{keep}\PY{o}{=}\PY{k+kc}{False}\PY{p}{)}\PY{p}{]}
\end{Verbatim}


\begin{Verbatim}[commandchars=\\\{\}]
{\color{outcolor}Out[{\color{outcolor}13}]:}       user\_id                   timestamp      group landing\_page  converted
         1899   773192  2017-01-09 05:37:58.781806  treatment     new\_page          0
         2893   773192  2017-01-14 02:55:59.590927  treatment     new\_page          0
\end{Verbatim}
            
    \begin{enumerate}
\def\labelenumi{\alph{enumi}.}
\setcounter{enumi}{3}
\tightlist
\item
  Remove \textbf{one} of the rows with a duplicate \textbf{user\_id},
  but keep your dataframe as \textbf{df2}.
\end{enumerate}

    \begin{Verbatim}[commandchars=\\\{\}]
{\color{incolor}In [{\color{incolor}14}]:} \PY{n}{df2} \PY{o}{=} \PY{n}{df2}\PY{o}{.}\PY{n}{drop}\PY{p}{(}\PY{p}{[}\PY{l+m+mi}{2893}\PY{p}{]}\PY{p}{)}
\end{Verbatim}


    \texttt{4.} Use \textbf{df2} in the below cells to answer the quiz
questions related to \textbf{Quiz 4} in the classroom.

\begin{enumerate}
\def\labelenumi{\alph{enumi}.}
\tightlist
\item
  What is the probability of an individual converting regardless of the
  page they receive?
\end{enumerate}

    \begin{Verbatim}[commandchars=\\\{\}]
{\color{incolor}In [{\color{incolor}15}]:} \PY{n}{conv\PYZus{}prob} \PY{o}{=} \PY{n}{df2}\PY{p}{[}\PY{l+s+s1}{\PYZsq{}}\PY{l+s+s1}{converted}\PY{l+s+s1}{\PYZsq{}}\PY{p}{]}\PY{o}{.}\PY{n}{sum}\PY{p}{(}\PY{p}{)} \PY{o}{/} \PY{n}{df2}\PY{p}{[}\PY{l+s+s1}{\PYZsq{}}\PY{l+s+s1}{converted}\PY{l+s+s1}{\PYZsq{}}\PY{p}{]}\PY{o}{.}\PY{n}{count}\PY{p}{(}\PY{p}{)}
         \PY{n+nb}{print}\PY{p}{(}\PY{l+s+s1}{\PYZsq{}}\PY{l+s+s1}{The probability of an individual converting regardless of page is }\PY{l+s+si}{\PYZob{}\PYZcb{}}\PY{l+s+s1}{\PYZsq{}}\PY{o}{.}\PY{n}{format}\PY{p}{(}\PY{n}{conv\PYZus{}prob}\PY{p}{)}\PY{p}{)}
\end{Verbatim}


    \begin{Verbatim}[commandchars=\\\{\}]
The probability of an individual converting regardless of page is 0.11959708724499628

    \end{Verbatim}

    \begin{enumerate}
\def\labelenumi{\alph{enumi}.}
\setcounter{enumi}{1}
\tightlist
\item
  Given that an individual was in the \texttt{control} group, what is
  the probability they converted?
\end{enumerate}

    \begin{Verbatim}[commandchars=\\\{\}]
{\color{incolor}In [{\color{incolor}16}]:} \PY{n}{control\PYZus{}prob} \PY{o}{=} \PY{n}{df2}\PY{o}{.}\PY{n}{query}\PY{p}{(}\PY{l+s+s1}{\PYZsq{}}\PY{l+s+s1}{group == }\PY{l+s+s1}{\PYZdq{}}\PY{l+s+s1}{control}\PY{l+s+s1}{\PYZdq{}}\PY{l+s+s1}{\PYZsq{}}\PY{p}{)}\PY{p}{[}\PY{l+s+s1}{\PYZsq{}}\PY{l+s+s1}{converted}\PY{l+s+s1}{\PYZsq{}}\PY{p}{]}\PY{o}{.}\PY{n}{sum}\PY{p}{(}\PY{p}{)} \PY{o}{/} \PY{n}{df2}\PY{o}{.}\PY{n}{query}\PY{p}{(}\PY{l+s+s1}{\PYZsq{}}\PY{l+s+s1}{group == }\PY{l+s+s1}{\PYZdq{}}\PY{l+s+s1}{control}\PY{l+s+s1}{\PYZdq{}}\PY{l+s+s1}{\PYZsq{}}\PY{p}{)}\PY{p}{[}\PY{l+s+s1}{\PYZsq{}}\PY{l+s+s1}{converted}\PY{l+s+s1}{\PYZsq{}}\PY{p}{]}\PY{o}{.}\PY{n}{count}\PY{p}{(}\PY{p}{)}
         \PY{n+nb}{print}\PY{p}{(}\PY{l+s+s1}{\PYZsq{}}\PY{l+s+s1}{The probability of an individual converting if they were in the control group is }\PY{l+s+si}{\PYZob{}\PYZcb{}}\PY{l+s+s1}{\PYZsq{}}\PY{o}{.}\PY{n}{format}\PY{p}{(}\PY{n}{control\PYZus{}prob}\PY{p}{)}\PY{p}{)}
\end{Verbatim}


    \begin{Verbatim}[commandchars=\\\{\}]
The probability of an individual converting if they were in the control group is 0.1203863045004612

    \end{Verbatim}

    \begin{enumerate}
\def\labelenumi{\alph{enumi}.}
\setcounter{enumi}{2}
\tightlist
\item
  Given that an individual was in the \texttt{treatment} group, what is
  the probability they converted?
\end{enumerate}

    \begin{Verbatim}[commandchars=\\\{\}]
{\color{incolor}In [{\color{incolor}17}]:} \PY{n}{treatment\PYZus{}prob} \PY{o}{=} \PY{n}{df2}\PY{o}{.}\PY{n}{query}\PY{p}{(}\PY{l+s+s1}{\PYZsq{}}\PY{l+s+s1}{group == }\PY{l+s+s1}{\PYZdq{}}\PY{l+s+s1}{treatment}\PY{l+s+s1}{\PYZdq{}}\PY{l+s+s1}{\PYZsq{}}\PY{p}{)}\PY{p}{[}\PY{l+s+s1}{\PYZsq{}}\PY{l+s+s1}{converted}\PY{l+s+s1}{\PYZsq{}}\PY{p}{]}\PY{o}{.}\PY{n}{sum}\PY{p}{(}\PY{p}{)} \PY{o}{/} \PY{n}{df2}\PY{o}{.}\PY{n}{query}\PY{p}{(}\PY{l+s+s1}{\PYZsq{}}\PY{l+s+s1}{group == }\PY{l+s+s1}{\PYZdq{}}\PY{l+s+s1}{treatment}\PY{l+s+s1}{\PYZdq{}}\PY{l+s+s1}{\PYZsq{}}\PY{p}{)}\PY{p}{[}\PY{l+s+s1}{\PYZsq{}}\PY{l+s+s1}{converted}\PY{l+s+s1}{\PYZsq{}}\PY{p}{]}\PY{o}{.}\PY{n}{count}\PY{p}{(}\PY{p}{)}
         \PY{n+nb}{print}\PY{p}{(}\PY{l+s+s1}{\PYZsq{}}\PY{l+s+s1}{The probability of an individual converting if they were shown the treatment group is }\PY{l+s+si}{\PYZob{}\PYZcb{}}\PY{l+s+s1}{\PYZsq{}}\PY{o}{.}\PY{n}{format}\PY{p}{(}\PY{n}{treatment\PYZus{}prob}\PY{p}{)}\PY{p}{)}
\end{Verbatim}


    \begin{Verbatim}[commandchars=\\\{\}]
The probability of an individual converting if they were shown the treatment group is 0.11880806551510564

    \end{Verbatim}

    \begin{enumerate}
\def\labelenumi{\alph{enumi}.}
\setcounter{enumi}{3}
\tightlist
\item
  What is the probability that an individual received the new page?
\end{enumerate}

    \begin{Verbatim}[commandchars=\\\{\}]
{\color{incolor}In [{\color{incolor}18}]:} \PY{n}{prob\PYZus{}new} \PY{o}{=} \PY{n}{df2}\PY{o}{.}\PY{n}{query}\PY{p}{(}\PY{l+s+s1}{\PYZsq{}}\PY{l+s+s1}{landing\PYZus{}page == }\PY{l+s+s1}{\PYZdq{}}\PY{l+s+s1}{new\PYZus{}page}\PY{l+s+s1}{\PYZdq{}}\PY{l+s+s1}{\PYZsq{}}\PY{p}{)}\PY{p}{[}\PY{l+s+s1}{\PYZsq{}}\PY{l+s+s1}{user\PYZus{}id}\PY{l+s+s1}{\PYZsq{}}\PY{p}{]}\PY{o}{.}\PY{n}{count}\PY{p}{(}\PY{p}{)} \PY{o}{/} \PY{n}{df2}\PY{p}{[}\PY{l+s+s1}{\PYZsq{}}\PY{l+s+s1}{user\PYZus{}id}\PY{l+s+s1}{\PYZsq{}}\PY{p}{]}\PY{o}{.}\PY{n}{count}\PY{p}{(}\PY{p}{)}
         \PY{n+nb}{print}\PY{p}{(}\PY{l+s+s1}{\PYZsq{}}\PY{l+s+s1}{The probability of an individual receiving the new page is }\PY{l+s+si}{\PYZob{}\PYZcb{}}\PY{l+s+s1}{\PYZsq{}}\PY{o}{.}\PY{n}{format}\PY{p}{(}\PY{n}{prob\PYZus{}new}\PY{p}{)}\PY{p}{)}
\end{Verbatim}


    \begin{Verbatim}[commandchars=\\\{\}]
The probability of an individual receiving the new page is 0.5000619442226688

    \end{Verbatim}

    \begin{enumerate}
\def\labelenumi{\alph{enumi}.}
\setcounter{enumi}{4}
\tightlist
\item
  Consider your results from a. through d. above, and explain below
  whether you think there is sufficient evidence to say that the new
  treatment page leads to more conversions.
\end{enumerate}

    \textbf{From the answers to the questions we can see that 50\% of those
tested recieved the new\_page vs.~the old\_page. We can also see that of
all users in the data collected that 11.9\% of them converted. When we
look at just the test group and the control group and how they converted
we can't really see a huge difference in conversion rates with the
control group converting at 12.0\% and the treatment group converting at
11.9\%. From this data it wouldn't be seem that the new treatment to the
page brings more conversions}

     \#\#\# Part II - A/B Test

Notice that because of the time stamp associated with each event, you
could technically run a hypothesis test continuously as each observation
was observed.

However, then the hard question is do you stop as soon as one page is
considered significantly better than another or does it need to happen
consistently for a certain amount of time? How long do you run to render
a decision that neither page is better than another?

These questions are the difficult parts associated with A/B tests in
general.

\texttt{1.} For now, consider you need to make the decision just based
on all the data provided. If you want to assume that the old page is
better unless the new page proves to be definitely better at a Type I
error rate of 5\%, what should your null and alternative hypotheses be?
You can state your hypothesis in terms of words or in terms of
\textbf{\(p_{old}\)} and \textbf{\(p_{new}\)}, which are the converted
rates for the old and new pages.

    \begin{quote}
\(H_{0} : \textbf{p}_{\textbf{old}} = \textbf{p}_{\textbf{new}}\)
\end{quote}

\begin{quote}
\(H_{1} : \textbf{p}_{\textbf{new}} > \textbf{p}_{\textbf{old}}\)
\end{quote}

    \texttt{2.} Assume under the null hypothesis, \(p_{new}\) and
\(p_{old}\) both have ``true'' success rates equal to the
\textbf{converted} success rate regardless of page - that is \(p_{new}\)
and \(p_{old}\) are equal. Furthermore, assume they are equal to the
\textbf{converted} rate in \textbf{ab\_data.csv} regardless of the page.

Use a sample size for each page equal to the ones in
\textbf{ab\_data.csv}.

Perform the sampling distribution for the difference in
\textbf{converted} between the two pages over 10,000 iterations of
calculating an estimate from the null.

Use the cells below to provide the necessary parts of this simulation.
If this doesn't make complete sense right now, don't worry - you are
going to work through the problems below to complete this problem. You
can use \textbf{Quiz 5} in the classroom to make sure you are on the
right track.

    \begin{enumerate}
\def\labelenumi{\alph{enumi}.}
\tightlist
\item
  What is the \textbf{convert rate} for \(p_{new}\) under the null?
\end{enumerate}

    \begin{Verbatim}[commandchars=\\\{\}]
{\color{incolor}In [{\color{incolor}19}]:} \PY{c+c1}{\PYZsh{}Simulate 10000 tests, we can use the conv\PYZus{}prob because we are saying that pnew and pold equal each}
         \PY{c+c1}{\PYZsh{}other under the null. This allows us to use the overall conversion rate.}
         \PY{n}{n\PYZus{}new} \PY{o}{=} \PY{n}{df2}\PY{o}{.}\PY{n}{query}\PY{p}{(}\PY{l+s+s1}{\PYZsq{}}\PY{l+s+s1}{landing\PYZus{}page == }\PY{l+s+s1}{\PYZdq{}}\PY{l+s+s1}{new\PYZus{}page}\PY{l+s+s1}{\PYZdq{}}\PY{l+s+s1}{\PYZsq{}}\PY{p}{)}\PY{o}{.}\PY{n}{count}\PY{p}{(}\PY{p}{)}\PY{p}{[}\PY{l+m+mi}{0}\PY{p}{]}
         \PY{n}{p\PYZus{}new} \PY{o}{=} \PY{n}{np}\PY{o}{.}\PY{n}{random}\PY{o}{.}\PY{n}{binomial}\PY{p}{(}\PY{n}{n\PYZus{}new}\PY{p}{,}\PY{n}{conv\PYZus{}prob}\PY{p}{,}\PY{l+m+mi}{10000}\PY{p}{)}\PY{o}{/}\PY{n}{n\PYZus{}new}
         \PY{n}{p\PYZus{}new}\PY{o}{.}\PY{n}{mean}\PY{p}{(}\PY{p}{)}
\end{Verbatim}


\begin{Verbatim}[commandchars=\\\{\}]
{\color{outcolor}Out[{\color{outcolor}19}]:} 0.11959796504025876
\end{Verbatim}
            
    \begin{enumerate}
\def\labelenumi{\alph{enumi}.}
\setcounter{enumi}{1}
\tightlist
\item
  What is the \textbf{convert rate} for \(p_{old}\) under the null? 
\end{enumerate}

    \begin{Verbatim}[commandchars=\\\{\}]
{\color{incolor}In [{\color{incolor}20}]:} \PY{c+c1}{\PYZsh{}Simulate 10000 tests, we can use the conv\PYZus{}prob because we are saying that pnew and pold equal each}
         \PY{c+c1}{\PYZsh{}other under the null. This allows us to use the overall conversion rate.}
         \PY{n}{n\PYZus{}old} \PY{o}{=} \PY{n}{df2}\PY{o}{.}\PY{n}{query}\PY{p}{(}\PY{l+s+s1}{\PYZsq{}}\PY{l+s+s1}{landing\PYZus{}page == }\PY{l+s+s1}{\PYZdq{}}\PY{l+s+s1}{old\PYZus{}page}\PY{l+s+s1}{\PYZdq{}}\PY{l+s+s1}{\PYZsq{}}\PY{p}{)}\PY{o}{.}\PY{n}{count}\PY{p}{(}\PY{p}{)}\PY{p}{[}\PY{l+m+mi}{0}\PY{p}{]}
         \PY{n}{p\PYZus{}old} \PY{o}{=} \PY{n}{np}\PY{o}{.}\PY{n}{random}\PY{o}{.}\PY{n}{binomial}\PY{p}{(}\PY{n}{n\PYZus{}old}\PY{p}{,}\PY{n}{conv\PYZus{}prob}\PY{p}{,}\PY{l+m+mi}{10000}\PY{p}{)}\PY{o}{/}\PY{n}{n\PYZus{}old}
         \PY{n}{p\PYZus{}old}\PY{o}{.}\PY{n}{mean}\PY{p}{(}\PY{p}{)}
\end{Verbatim}


\begin{Verbatim}[commandchars=\\\{\}]
{\color{outcolor}Out[{\color{outcolor}20}]:} 0.11959628839296779
\end{Verbatim}
            
    \begin{enumerate}
\def\labelenumi{\alph{enumi}.}
\setcounter{enumi}{2}
\tightlist
\item
  What is \(n_{new}\)?
\end{enumerate}

    \begin{Verbatim}[commandchars=\\\{\}]
{\color{incolor}In [{\color{incolor}30}]:} \PY{n}{n\PYZus{}new}
\end{Verbatim}


\begin{Verbatim}[commandchars=\\\{\}]
{\color{outcolor}Out[{\color{outcolor}30}]:} 145310
\end{Verbatim}
            
    \begin{enumerate}
\def\labelenumi{\alph{enumi}.}
\setcounter{enumi}{3}
\tightlist
\item
  What is \(n_{old}\)?
\end{enumerate}

    \begin{Verbatim}[commandchars=\\\{\}]
{\color{incolor}In [{\color{incolor}31}]:} \PY{n}{n\PYZus{}old}
\end{Verbatim}


\begin{Verbatim}[commandchars=\\\{\}]
{\color{outcolor}Out[{\color{outcolor}31}]:} 145274
\end{Verbatim}
            
    \begin{enumerate}
\def\labelenumi{\alph{enumi}.}
\setcounter{enumi}{4}
\tightlist
\item
  Simulate \(n_{new}\) transactions with a convert rate of \(p_{new}\)
  under the null. Store these \(n_{new}\) 1's and 0's in
  \textbf{new\_page\_converted}.
\end{enumerate}

    \begin{Verbatim}[commandchars=\\\{\}]
{\color{incolor}In [{\color{incolor}34}]:} \PY{n}{new\PYZus{}page\PYZus{}converted} \PY{o}{=} \PY{n}{np}\PY{o}{.}\PY{n}{random}\PY{o}{.}\PY{n}{choice}\PY{p}{(}\PY{p}{[}\PY{l+m+mi}{0}\PY{p}{,}\PY{l+m+mi}{1}\PY{p}{]}\PY{p}{,} \PY{n}{n\PYZus{}new}\PY{p}{,} \PY{n}{conv\PYZus{}prob}\PY{p}{)}
         \PY{n}{new\PYZus{}page\PYZus{}converted}\PY{o}{.}\PY{n}{mean}\PY{p}{(}\PY{p}{)}
\end{Verbatim}


\begin{Verbatim}[commandchars=\\\{\}]
{\color{outcolor}Out[{\color{outcolor}34}]:} 0.50037161929667606
\end{Verbatim}
            
    \begin{enumerate}
\def\labelenumi{\alph{enumi}.}
\setcounter{enumi}{5}
\tightlist
\item
  Simulate \(n_{old}\) transactions with a convert rate of \(p_{old}\)
  under the null. Store these \(n_{old}\) 1's and 0's in
  \textbf{old\_page\_converted}.
\end{enumerate}

    \begin{Verbatim}[commandchars=\\\{\}]
{\color{incolor}In [{\color{incolor}35}]:} \PY{n}{old\PYZus{}page\PYZus{}converted} \PY{o}{=} \PY{n}{np}\PY{o}{.}\PY{n}{random}\PY{o}{.}\PY{n}{choice}\PY{p}{(}\PY{p}{[}\PY{l+m+mi}{0}\PY{p}{,}\PY{l+m+mi}{1}\PY{p}{]}\PY{p}{,} \PY{n}{n\PYZus{}old}\PY{p}{,} \PY{n}{conv\PYZus{}prob}\PY{p}{)}
         \PY{n}{old\PYZus{}page\PYZus{}converted}\PY{o}{.}\PY{n}{mean}\PY{p}{(}\PY{p}{)}
\end{Verbatim}


\begin{Verbatim}[commandchars=\\\{\}]
{\color{outcolor}Out[{\color{outcolor}35}]:} 0.50092927846689705
\end{Verbatim}
            
    \begin{enumerate}
\def\labelenumi{\alph{enumi}.}
\setcounter{enumi}{6}
\tightlist
\item
  Find \(p_{new}\) - \(p_{old}\) for your simulated values from part (e)
  and (f).
\end{enumerate}

    \begin{Verbatim}[commandchars=\\\{\}]
{\color{incolor}In [{\color{incolor}36}]:} \PY{n}{new\PYZus{}page\PYZus{}converted}\PY{o}{.}\PY{n}{mean}\PY{p}{(}\PY{p}{)} \PY{o}{\PYZhy{}} \PY{n}{old\PYZus{}page\PYZus{}converted}\PY{o}{.}\PY{n}{mean}\PY{p}{(}\PY{p}{)}
\end{Verbatim}


\begin{Verbatim}[commandchars=\\\{\}]
{\color{outcolor}Out[{\color{outcolor}36}]:} -0.0005576591702209921
\end{Verbatim}
            
    \begin{enumerate}
\def\labelenumi{\alph{enumi}.}
\setcounter{enumi}{7}
\tightlist
\item
  Simulate 10,000 \(p_{new}\) - \(p_{old}\) values using this same
  process similarly to the one you calculated in parts \textbf{a.
  through g.} above. Store all 10,000 values in a numpy array called
  \textbf{p\_diffs}.
\end{enumerate}

    \begin{Verbatim}[commandchars=\\\{\}]
{\color{incolor}In [{\color{incolor}37}]:} \PY{n}{p\PYZus{}diffs} \PY{o}{=} \PY{p}{[}\PY{p}{]}
         \PY{k}{for} \PY{n}{\PYZus{}} \PY{o+ow}{in} \PY{n+nb}{range}\PY{p}{(}\PY{l+m+mi}{10000}\PY{p}{)}\PY{p}{:}
             \PY{n}{new\PYZus{}page\PYZus{}converted} \PY{o}{=} \PY{n}{np}\PY{o}{.}\PY{n}{random}\PY{o}{.}\PY{n}{choice}\PY{p}{(}\PY{p}{[}\PY{l+m+mi}{0}\PY{p}{,}\PY{l+m+mi}{1}\PY{p}{]}\PY{p}{,} \PY{n}{n\PYZus{}new}\PY{p}{,} \PY{n}{conv\PYZus{}prob}\PY{p}{)}
             \PY{n}{old\PYZus{}page\PYZus{}converted} \PY{o}{=} \PY{n}{np}\PY{o}{.}\PY{n}{random}\PY{o}{.}\PY{n}{choice}\PY{p}{(}\PY{p}{[}\PY{l+m+mi}{0}\PY{p}{,}\PY{l+m+mi}{1}\PY{p}{]}\PY{p}{,} \PY{n}{n\PYZus{}old}\PY{p}{,} \PY{n}{conv\PYZus{}prob}\PY{p}{)}
             \PY{n}{p\PYZus{}diffs}\PY{o}{.}\PY{n}{append}\PY{p}{(}\PY{n}{new\PYZus{}page\PYZus{}converted}\PY{o}{.}\PY{n}{mean}\PY{p}{(}\PY{p}{)} \PY{o}{\PYZhy{}} \PY{n}{old\PYZus{}page\PYZus{}converted}\PY{o}{.}\PY{n}{mean}\PY{p}{(}\PY{p}{)}\PY{p}{)}
         
         \PY{n}{np}\PY{o}{.}\PY{n}{asanyarray}\PY{p}{(}\PY{n}{p\PYZus{}diffs}\PY{p}{)}
\end{Verbatim}


\begin{Verbatim}[commandchars=\\\{\}]
{\color{outcolor}Out[{\color{outcolor}37}]:} array([-0.00200953, -0.00174792,  0.00037894, {\ldots}, -0.00028218,
                 0.00315919,  0.00112181])
\end{Verbatim}
            
    \begin{enumerate}
\def\labelenumi{\roman{enumi}.}
\tightlist
\item
  Plot a histogram of the \textbf{p\_diffs}. Does this plot look like
  what you expected? Use the matching problem in the classroom to assure
  you fully understand what was computed here.
\end{enumerate}

    \begin{Verbatim}[commandchars=\\\{\}]
{\color{incolor}In [{\color{incolor}38}]:} \PY{n}{plt}\PY{o}{.}\PY{n}{hist}\PY{p}{(}\PY{n}{p\PYZus{}diffs}\PY{p}{)}
\end{Verbatim}


\begin{Verbatim}[commandchars=\\\{\}]
{\color{outcolor}Out[{\color{outcolor}38}]:} (array([    5.,    74.,   448.,  1570.,  2941.,  2956.,  1516.,   431.,
                    53.,     6.]),
          array([ -7.61217496e-03,  -6.07734720e-03,  -4.54251944e-03,
                  -3.00769168e-03,  -1.47286392e-03,   6.19638354e-05,
                   1.59679159e-03,   3.13161935e-03,   4.66644711e-03,
                   6.20127487e-03,   7.73610263e-03]),
          <a list of 10 Patch objects>)
\end{Verbatim}
            
    \begin{center}
    \adjustimage{max size={0.9\linewidth}{0.9\paperheight}}{output_57_1.png}
    \end{center}
    { \hspace*{\fill} \\}
    
    \begin{enumerate}
\def\labelenumi{\alph{enumi}.}
\setcounter{enumi}{9}
\tightlist
\item
  What proportion of the \textbf{p\_diffs} are greater than the actual
  difference observed in \textbf{ab\_data.csv}?
\end{enumerate}

    \begin{Verbatim}[commandchars=\\\{\}]
{\color{incolor}In [{\color{incolor}39}]:} \PY{p}{(}\PY{n}{p\PYZus{}diffs} \PY{o}{\PYZgt{}} \PY{n}{treatment\PYZus{}prob} \PY{o}{\PYZhy{}} \PY{n}{control\PYZus{}prob}\PY{p}{)}\PY{o}{.}\PY{n}{mean}\PY{p}{(}\PY{p}{)}
\end{Verbatim}


\begin{Verbatim}[commandchars=\\\{\}]
{\color{outcolor}Out[{\color{outcolor}39}]:} 0.80520000000000003
\end{Verbatim}
            
    \begin{enumerate}
\def\labelenumi{\alph{enumi}.}
\setcounter{enumi}{10}
\tightlist
\item
  In words, explain what you just computed in part \textbf{j.} What is
  this value called in scientific studies? What does this value mean in
  terms of whether or not there is a difference between the new and old
  pages?
\end{enumerate}

    \textbf{What we've calculated is the difference in p values using the
central limit theorem and compared that to the difference we saw in the
single values generated in the beginning of the project. We can see that
by doing 1,000s of trials and comparing them to orignal values we're
able to calculate a probablity of rejecting or not rejecting the null
hypothosis. With a vlaue of 0.7987 we would have to not reject the null
hypothosis. i.e.~We have to conclude that p\_new and p\_old are equal to
each other.}

    \begin{enumerate}
\def\labelenumi{\alph{enumi}.}
\setcounter{enumi}{11}
\tightlist
\item
  We could also use a built-in to achieve similar results. Though using
  the built-in might be easier to code, the above portions are a
  walkthrough of the ideas that are critical to correctly thinking about
  statistical significance. Fill in the below to calculate the number of
  conversions for each page, as well as the number of individuals who
  received each page. Let \texttt{n\_old} and \texttt{n\_new} refer the
  the number of rows associated with the old page and new pages,
  respectively.
\end{enumerate}

    \begin{Verbatim}[commandchars=\\\{\}]
{\color{incolor}In [{\color{incolor}40}]:} \PY{k+kn}{import} \PY{n+nn}{statsmodels}\PY{n+nn}{.}\PY{n+nn}{api} \PY{k}{as} \PY{n+nn}{sm}
         
         \PY{n}{convert\PYZus{}old} \PY{o}{=} \PY{n}{df2}\PY{o}{.}\PY{n}{query}\PY{p}{(}\PY{l+s+s1}{\PYZsq{}}\PY{l+s+s1}{group == }\PY{l+s+s1}{\PYZdq{}}\PY{l+s+s1}{control}\PY{l+s+s1}{\PYZdq{}}\PY{l+s+s1}{\PYZsq{}}\PY{p}{)}\PY{p}{[}\PY{l+s+s1}{\PYZsq{}}\PY{l+s+s1}{converted}\PY{l+s+s1}{\PYZsq{}}\PY{p}{]}\PY{o}{.}\PY{n}{sum}\PY{p}{(}\PY{p}{)}
         \PY{n}{convert\PYZus{}new} \PY{o}{=} \PY{n}{df2}\PY{o}{.}\PY{n}{query}\PY{p}{(}\PY{l+s+s1}{\PYZsq{}}\PY{l+s+s1}{group == }\PY{l+s+s1}{\PYZdq{}}\PY{l+s+s1}{treatment}\PY{l+s+s1}{\PYZdq{}}\PY{l+s+s1}{\PYZsq{}}\PY{p}{)}\PY{p}{[}\PY{l+s+s1}{\PYZsq{}}\PY{l+s+s1}{converted}\PY{l+s+s1}{\PYZsq{}}\PY{p}{]}\PY{o}{.}\PY{n}{sum}\PY{p}{(}\PY{p}{)}
         \PY{n}{n\PYZus{}old} \PY{o}{=} \PY{n}{df2}\PY{o}{.}\PY{n}{query}\PY{p}{(}\PY{l+s+s1}{\PYZsq{}}\PY{l+s+s1}{group == }\PY{l+s+s1}{\PYZdq{}}\PY{l+s+s1}{control}\PY{l+s+s1}{\PYZdq{}}\PY{l+s+s1}{\PYZsq{}}\PY{p}{)}\PY{o}{.}\PY{n}{count}\PY{p}{(}\PY{p}{)}\PY{p}{[}\PY{l+m+mi}{0}\PY{p}{]}
         \PY{n}{n\PYZus{}new} \PY{o}{=} \PY{n}{df2}\PY{o}{.}\PY{n}{query}\PY{p}{(}\PY{l+s+s1}{\PYZsq{}}\PY{l+s+s1}{group == }\PY{l+s+s1}{\PYZdq{}}\PY{l+s+s1}{treatment}\PY{l+s+s1}{\PYZdq{}}\PY{l+s+s1}{\PYZsq{}}\PY{p}{)}\PY{o}{.}\PY{n}{count}\PY{p}{(}\PY{p}{)}\PY{p}{[}\PY{l+m+mi}{0}\PY{p}{]}
\end{Verbatim}


    \begin{Verbatim}[commandchars=\\\{\}]
C:\textbackslash{}Users\textbackslash{}ryanj\textbackslash{}AppData\textbackslash{}Local\textbackslash{}Continuum\textbackslash{}anaconda3\textbackslash{}lib\textbackslash{}site-packages\textbackslash{}statsmodels\textbackslash{}compat\textbackslash{}pandas.py:56: FutureWarning: The pandas.core.datetools module is deprecated and will be removed in a future version. Please use the pandas.tseries module instead.
  from pandas.core import datetools

    \end{Verbatim}

    \begin{enumerate}
\def\labelenumi{\alph{enumi}.}
\setcounter{enumi}{12}
\tightlist
\item
  Now use \texttt{stats.proportions\_ztest} to compute your test
  statistic and p-value.
  \href{http://knowledgetack.com/python/statsmodels/proportions_ztest/}{Here}
  is a helpful link on using the built in.
\end{enumerate}

    \begin{Verbatim}[commandchars=\\\{\}]
{\color{incolor}In [{\color{incolor}41}]:} \PY{n}{z\PYZus{}score}\PY{p}{,} \PY{n}{p\PYZus{}value} \PY{o}{=} \PY{n}{sm}\PY{o}{.}\PY{n}{stats}\PY{o}{.}\PY{n}{proportions\PYZus{}ztest}\PY{p}{(}\PY{p}{[}\PY{n}{convert\PYZus{}old}\PY{p}{,} \PY{n}{convert\PYZus{}new}\PY{p}{]}\PY{p}{,} \PY{p}{[}\PY{n}{n\PYZus{}old}\PY{p}{,} \PY{n}{n\PYZus{}new}\PY{p}{]}\PY{p}{)}
         \PY{n+nb}{print}\PY{p}{(}\PY{n}{z\PYZus{}score}\PY{p}{,} \PY{n}{p\PYZus{}value}\PY{p}{)}
\end{Verbatim}


    \begin{Verbatim}[commandchars=\\\{\}]
1.31092419842 0.189883374482

    \end{Verbatim}

    \begin{enumerate}
\def\labelenumi{\alph{enumi}.}
\setcounter{enumi}{13}
\tightlist
\item
  What do the z-score and p-value you computed in the previous question
  mean for the conversion rates of the old and new pages? Do they agree
  with the findings in parts \textbf{j.} and \textbf{k.}?
\end{enumerate}

    \textbf{The z\_score and p\_value obtained from this proportions test
confirms what we have already concluded in sections j and k. Having a
z\_score \textgreater{} 1.96 or \textless{} -1.96 says that your results
are significant and you should reject the null, at the same time having
a p\_value \textless{} 0.05 says we should reject the null. With values
of 1.31 and 0.1898 respectively we are well below the thereshold of
being allowed to reject the null hypothosis, and thus are required to
keep it as found out in previous steps.}

     \#\#\# Part III - A regression approach

\texttt{1.} In this final part, you will see that the result you
acheived in the previous A/B test can also be acheived by performing
regression.

\begin{enumerate}
\def\labelenumi{\alph{enumi}.}
\tightlist
\item
  Since each row is either a conversion or no conversion, what type of
  regression should you be performing in this case?
\end{enumerate}

    \textbf{We'd want to use a logistic regression with having an
``attribute'' type of dependent variable.}

    \begin{enumerate}
\def\labelenumi{\alph{enumi}.}
\setcounter{enumi}{1}
\tightlist
\item
  The goal is to use \textbf{statsmodels} to fit the regression model
  you specified in part \textbf{a.} to see if there is a significant
  difference in conversion based on which page a customer receives.
  However, you first need to create a column for the intercept, and
  create a dummy variable column for which page each user received. Add
  an \textbf{intercept} column, as well as an \textbf{ab\_page} column,
  which is 1 when an individual receives the \textbf{treatment} and 0 if
  \textbf{control}.
\end{enumerate}

    \begin{Verbatim}[commandchars=\\\{\}]
{\color{incolor}In [{\color{incolor}63}]:} \PY{n}{df2}\PY{p}{[}\PY{l+s+s1}{\PYZsq{}}\PY{l+s+s1}{intercept}\PY{l+s+s1}{\PYZsq{}}\PY{p}{]} \PY{o}{=} \PY{l+m+mi}{1}
         \PY{n}{df2}\PY{p}{[}\PY{p}{[}\PY{l+s+s1}{\PYZsq{}}\PY{l+s+s1}{control\PYZus{}page}\PY{l+s+s1}{\PYZsq{}}\PY{p}{,} \PY{l+s+s1}{\PYZsq{}}\PY{l+s+s1}{treatment\PYZus{}page}\PY{l+s+s1}{\PYZsq{}}\PY{p}{]}\PY{p}{]} \PY{o}{=} \PY{n}{pd}\PY{o}{.}\PY{n}{get\PYZus{}dummies}\PY{p}{(}\PY{n}{df2}\PY{p}{[}\PY{l+s+s1}{\PYZsq{}}\PY{l+s+s1}{group}\PY{l+s+s1}{\PYZsq{}}\PY{p}{]}\PY{p}{)}
\end{Verbatim}


    \begin{enumerate}
\def\labelenumi{\alph{enumi}.}
\setcounter{enumi}{2}
\tightlist
\item
  Use \textbf{statsmodels} to import your regression model. Instantiate
  the model, and fit the model using the two columns you created in part
  \textbf{b.} to predict whether or not an individual converts.
\end{enumerate}

    \begin{Verbatim}[commandchars=\\\{\}]
{\color{incolor}In [{\color{incolor}65}]:} \PY{n}{log\PYZus{}mod} \PY{o}{=} \PY{n}{sm}\PY{o}{.}\PY{n}{Logit}\PY{p}{(}\PY{n}{df2}\PY{p}{[}\PY{l+s+s1}{\PYZsq{}}\PY{l+s+s1}{converted}\PY{l+s+s1}{\PYZsq{}}\PY{p}{]}\PY{p}{,} \PY{n}{df2}\PY{p}{[}\PY{p}{[}\PY{l+s+s1}{\PYZsq{}}\PY{l+s+s1}{intercept}\PY{l+s+s1}{\PYZsq{}}\PY{p}{,} \PY{l+s+s1}{\PYZsq{}}\PY{l+s+s1}{treatment\PYZus{}page}\PY{l+s+s1}{\PYZsq{}}\PY{p}{]}\PY{p}{]}\PY{p}{)}
         \PY{n}{results} \PY{o}{=} \PY{n}{log\PYZus{}mod}\PY{o}{.}\PY{n}{fit}\PY{p}{(}\PY{p}{)}
\end{Verbatim}


    \begin{Verbatim}[commandchars=\\\{\}]
Optimization terminated successfully.
         Current function value: 0.366118
         Iterations 6

    \end{Verbatim}

    \begin{enumerate}
\def\labelenumi{\alph{enumi}.}
\setcounter{enumi}{3}
\tightlist
\item
  Provide the summary of your model below, and use it as necessary to
  answer the following questions.
\end{enumerate}

    \begin{Verbatim}[commandchars=\\\{\}]
{\color{incolor}In [{\color{incolor}66}]:} \PY{n}{results}\PY{o}{.}\PY{n}{summary}\PY{p}{(}\PY{p}{)}
\end{Verbatim}


\begin{Verbatim}[commandchars=\\\{\}]
{\color{outcolor}Out[{\color{outcolor}66}]:} <class 'statsmodels.iolib.summary.Summary'>
         """
                                    Logit Regression Results                           
         ==============================================================================
         Dep. Variable:              converted   No. Observations:               290584
         Model:                          Logit   Df Residuals:                   290582
         Method:                           MLE   Df Model:                            1
         Date:                Wed, 21 Nov 2018   Pseudo R-squ.:               8.077e-06
         Time:                        13:32:38   Log-Likelihood:            -1.0639e+05
         converged:                       True   LL-Null:                   -1.0639e+05
                                                 LLR p-value:                    0.1899
         ==================================================================================
                              coef    std err          z      P>|z|      [0.025      0.975]
         ----------------------------------------------------------------------------------
         intercept         -1.9888      0.008   -246.669      0.000      -2.005      -1.973
         treatment\_page    -0.0150      0.011     -1.311      0.190      -0.037       0.007
         ==================================================================================
         """
\end{Verbatim}
            
    \begin{Verbatim}[commandchars=\\\{\}]
{\color{incolor}In [{\color{incolor}67}]:} \PY{c+c1}{\PYZsh{} Odds Ratio \PYZhy{} The Odds of the treatment page being accepted over the control page}
         \PY{n}{np}\PY{o}{.}\PY{n}{exp}\PY{p}{(}\PY{n}{results}\PY{o}{.}\PY{n}{params}\PY{p}{)}
\end{Verbatim}


\begin{Verbatim}[commandchars=\\\{\}]
{\color{outcolor}Out[{\color{outcolor}67}]:} intercept         0.136863
         treatment\_page    0.985123
         dtype: float64
\end{Verbatim}
            
    \begin{enumerate}
\def\labelenumi{\alph{enumi}.}
\setcounter{enumi}{4}
\tightlist
\item
  What is the p-value associated with \textbf{ab\_page}? Why does it
  differ from the value you found in \textbf{Part II}? \textbf{Hint}:
  What are the null and alternative hypotheses associated with your
  regression model, and how do they compare to the null and alternative
  hypotheses in the \textbf{Part II}?
\end{enumerate}

    \textbf{The null and alternative hypotheses didn't change between
section 2 and 3. Our z-value is still showing that there is no
statistical difference at a value of -1.24. When we evaluate the odds
ratio we get a value of 0.985, which is basically 1 meaning that the
odds of the treatment being selected is the same as the control being
selected.}

    \begin{enumerate}
\def\labelenumi{\alph{enumi}.}
\setcounter{enumi}{5}
\tightlist
\item
  Now, you are considering other things that might influence whether or
  not an individual converts. Discuss why it is a good idea to consider
  other factors to add into your regression model. Are there any
  disadvantages to adding additional terms into your regression model?
\end{enumerate}

    \textbf{Considering other variables in your model can help to show you
if there are variables that correlate with your dependent. Having to few
might show that there is no correlation at all. However, you can get
into a situation where you can overfit your model showing something as
being significant when in actuallity it isn't.}

    \begin{enumerate}
\def\labelenumi{\alph{enumi}.}
\setcounter{enumi}{6}
\tightlist
\item
  Now along with testing if the conversion rate changes for different
  pages, also add an effect based on which country a user lives. You
  will need to read in the \textbf{countries.csv} dataset and merge
  together your datasets on the approporiate rows.
  \href{https://pandas.pydata.org/pandas-docs/stable/generated/pandas.DataFrame.join.html}{Here}
  are the docs for joining tables.
\end{enumerate}

Does it appear that country had an impact on conversion? Don't forget to
create dummy variables for these country columns - \textbf{Hint: You
will need two columns for the three dummy variables.} Provide the
statistical output as well as a written response to answer this
question.

    \begin{Verbatim}[commandchars=\\\{\}]
{\color{incolor}In [{\color{incolor}68}]:} \PY{n}{countries\PYZus{}df} \PY{o}{=} \PY{n}{pd}\PY{o}{.}\PY{n}{read\PYZus{}csv}\PY{p}{(}\PY{l+s+s1}{\PYZsq{}}\PY{l+s+s1}{./countries.csv}\PY{l+s+s1}{\PYZsq{}}\PY{p}{)}
         \PY{n}{df\PYZus{}new} \PY{o}{=} \PY{n}{countries\PYZus{}df}\PY{o}{.}\PY{n}{set\PYZus{}index}\PY{p}{(}\PY{l+s+s1}{\PYZsq{}}\PY{l+s+s1}{user\PYZus{}id}\PY{l+s+s1}{\PYZsq{}}\PY{p}{)}\PY{o}{.}\PY{n}{join}\PY{p}{(}\PY{n}{df2}\PY{o}{.}\PY{n}{set\PYZus{}index}\PY{p}{(}\PY{l+s+s1}{\PYZsq{}}\PY{l+s+s1}{user\PYZus{}id}\PY{l+s+s1}{\PYZsq{}}\PY{p}{)}\PY{p}{,} \PY{n}{how}\PY{o}{=}\PY{l+s+s1}{\PYZsq{}}\PY{l+s+s1}{inner}\PY{l+s+s1}{\PYZsq{}}\PY{p}{)}
         \PY{n}{df\PYZus{}new}\PY{o}{.}\PY{n}{head}\PY{p}{(}\PY{p}{)}
\end{Verbatim}


\begin{Verbatim}[commandchars=\\\{\}]
{\color{outcolor}Out[{\color{outcolor}68}]:}         country                   timestamp      group landing\_page  \textbackslash{}
         user\_id                                                               
         834778       UK  2017-01-14 23:08:43.304998    control     old\_page   
         928468       US  2017-01-23 14:44:16.387854  treatment     new\_page   
         822059       UK  2017-01-16 14:04:14.719771  treatment     new\_page   
         711597       UK  2017-01-22 03:14:24.763511    control     old\_page   
         710616       UK  2017-01-16 13:14:44.000513  treatment     new\_page   
         
                  converted  intercept  control\_page  treatment\_page  
         user\_id                                                      
         834778           0          1             1               0  
         928468           0          1             0               1  
         822059           1          1             0               1  
         711597           0          1             1               0  
         710616           0          1             0               1  
\end{Verbatim}
            
    \begin{Verbatim}[commandchars=\\\{\}]
{\color{incolor}In [{\color{incolor}77}]:} \PY{n}{df\PYZus{}new}\PY{o}{.}\PY{n}{country}\PY{o}{.}\PY{n}{unique}\PY{p}{(}\PY{p}{)}
\end{Verbatim}


\begin{Verbatim}[commandchars=\\\{\}]
{\color{outcolor}Out[{\color{outcolor}77}]:} array(['UK', 'US', 'CA'], dtype=object)
\end{Verbatim}
            
    \begin{Verbatim}[commandchars=\\\{\}]
{\color{incolor}In [{\color{incolor}79}]:} \PY{n}{df\PYZus{}new}\PY{p}{[}\PY{p}{[}\PY{l+s+s1}{\PYZsq{}}\PY{l+s+s1}{Country\PYZus{}CA}\PY{l+s+s1}{\PYZsq{}}\PY{p}{,} \PY{l+s+s1}{\PYZsq{}}\PY{l+s+s1}{Country\PYZus{}UK}\PY{l+s+s1}{\PYZsq{}}\PY{p}{,} \PY{l+s+s1}{\PYZsq{}}\PY{l+s+s1}{Country\PYZus{}US}\PY{l+s+s1}{\PYZsq{}}\PY{p}{]}\PY{p}{]} \PY{o}{=} \PY{n}{pd}\PY{o}{.}\PY{n}{get\PYZus{}dummies}\PY{p}{(}\PY{n}{df\PYZus{}new}\PY{o}{.}\PY{n}{country}\PY{p}{)}
\end{Verbatim}


    \begin{Verbatim}[commandchars=\\\{\}]
{\color{incolor}In [{\color{incolor}80}]:} \PY{n}{convert\PYZus{}old\PYZus{}CA} \PY{o}{=} \PY{n}{df\PYZus{}new}\PY{o}{.}\PY{n}{query}\PY{p}{(}\PY{l+s+s1}{\PYZsq{}}\PY{l+s+s1}{group == }\PY{l+s+s1}{\PYZdq{}}\PY{l+s+s1}{control}\PY{l+s+s1}{\PYZdq{}}\PY{l+s+s1}{ \PYZam{} country == }\PY{l+s+s1}{\PYZdq{}}\PY{l+s+s1}{CA}\PY{l+s+s1}{\PYZdq{}}\PY{l+s+s1}{\PYZsq{}}\PY{p}{)}\PY{p}{[}\PY{l+s+s1}{\PYZsq{}}\PY{l+s+s1}{converted}\PY{l+s+s1}{\PYZsq{}}\PY{p}{]}\PY{o}{.}\PY{n}{sum}\PY{p}{(}\PY{p}{)}
         \PY{n}{convert\PYZus{}old\PYZus{}UK} \PY{o}{=} \PY{n}{df\PYZus{}new}\PY{o}{.}\PY{n}{query}\PY{p}{(}\PY{l+s+s1}{\PYZsq{}}\PY{l+s+s1}{group == }\PY{l+s+s1}{\PYZdq{}}\PY{l+s+s1}{control}\PY{l+s+s1}{\PYZdq{}}\PY{l+s+s1}{ \PYZam{} country == }\PY{l+s+s1}{\PYZdq{}}\PY{l+s+s1}{UK}\PY{l+s+s1}{\PYZdq{}}\PY{l+s+s1}{\PYZsq{}}\PY{p}{)}\PY{p}{[}\PY{l+s+s1}{\PYZsq{}}\PY{l+s+s1}{converted}\PY{l+s+s1}{\PYZsq{}}\PY{p}{]}\PY{o}{.}\PY{n}{sum}\PY{p}{(}\PY{p}{)}
         \PY{n}{convert\PYZus{}old\PYZus{}US} \PY{o}{=} \PY{n}{df\PYZus{}new}\PY{o}{.}\PY{n}{query}\PY{p}{(}\PY{l+s+s1}{\PYZsq{}}\PY{l+s+s1}{group == }\PY{l+s+s1}{\PYZdq{}}\PY{l+s+s1}{control}\PY{l+s+s1}{\PYZdq{}}\PY{l+s+s1}{ \PYZam{} country == }\PY{l+s+s1}{\PYZdq{}}\PY{l+s+s1}{US}\PY{l+s+s1}{\PYZdq{}}\PY{l+s+s1}{\PYZsq{}}\PY{p}{)}\PY{p}{[}\PY{l+s+s1}{\PYZsq{}}\PY{l+s+s1}{converted}\PY{l+s+s1}{\PYZsq{}}\PY{p}{]}\PY{o}{.}\PY{n}{sum}\PY{p}{(}\PY{p}{)}
         \PY{n}{convert\PYZus{}new\PYZus{}CA} \PY{o}{=} \PY{n}{df\PYZus{}new}\PY{o}{.}\PY{n}{query}\PY{p}{(}\PY{l+s+s1}{\PYZsq{}}\PY{l+s+s1}{group == }\PY{l+s+s1}{\PYZdq{}}\PY{l+s+s1}{treatment}\PY{l+s+s1}{\PYZdq{}}\PY{l+s+s1}{ \PYZam{} country == }\PY{l+s+s1}{\PYZdq{}}\PY{l+s+s1}{CA}\PY{l+s+s1}{\PYZdq{}}\PY{l+s+s1}{\PYZsq{}}\PY{p}{)}\PY{p}{[}\PY{l+s+s1}{\PYZsq{}}\PY{l+s+s1}{converted}\PY{l+s+s1}{\PYZsq{}}\PY{p}{]}\PY{o}{.}\PY{n}{sum}\PY{p}{(}\PY{p}{)}
         \PY{n}{convert\PYZus{}new\PYZus{}UK} \PY{o}{=} \PY{n}{df\PYZus{}new}\PY{o}{.}\PY{n}{query}\PY{p}{(}\PY{l+s+s1}{\PYZsq{}}\PY{l+s+s1}{group == }\PY{l+s+s1}{\PYZdq{}}\PY{l+s+s1}{treatment}\PY{l+s+s1}{\PYZdq{}}\PY{l+s+s1}{ \PYZam{} country == }\PY{l+s+s1}{\PYZdq{}}\PY{l+s+s1}{UK}\PY{l+s+s1}{\PYZdq{}}\PY{l+s+s1}{\PYZsq{}}\PY{p}{)}\PY{p}{[}\PY{l+s+s1}{\PYZsq{}}\PY{l+s+s1}{converted}\PY{l+s+s1}{\PYZsq{}}\PY{p}{]}\PY{o}{.}\PY{n}{sum}\PY{p}{(}\PY{p}{)}
         \PY{n}{convert\PYZus{}new\PYZus{}US} \PY{o}{=} \PY{n}{df\PYZus{}new}\PY{o}{.}\PY{n}{query}\PY{p}{(}\PY{l+s+s1}{\PYZsq{}}\PY{l+s+s1}{group == }\PY{l+s+s1}{\PYZdq{}}\PY{l+s+s1}{treatment}\PY{l+s+s1}{\PYZdq{}}\PY{l+s+s1}{ \PYZam{} country == }\PY{l+s+s1}{\PYZdq{}}\PY{l+s+s1}{US}\PY{l+s+s1}{\PYZdq{}}\PY{l+s+s1}{\PYZsq{}}\PY{p}{)}\PY{p}{[}\PY{l+s+s1}{\PYZsq{}}\PY{l+s+s1}{converted}\PY{l+s+s1}{\PYZsq{}}\PY{p}{]}\PY{o}{.}\PY{n}{sum}\PY{p}{(}\PY{p}{)}
         \PY{n}{n\PYZus{}old\PYZus{}CA} \PY{o}{=} \PY{n}{df\PYZus{}new}\PY{o}{.}\PY{n}{query}\PY{p}{(}\PY{l+s+s1}{\PYZsq{}}\PY{l+s+s1}{group == }\PY{l+s+s1}{\PYZdq{}}\PY{l+s+s1}{control}\PY{l+s+s1}{\PYZdq{}}\PY{l+s+s1}{  \PYZam{} country == }\PY{l+s+s1}{\PYZdq{}}\PY{l+s+s1}{CA}\PY{l+s+s1}{\PYZdq{}}\PY{l+s+s1}{\PYZsq{}}\PY{p}{)}\PY{o}{.}\PY{n}{count}\PY{p}{(}\PY{p}{)}\PY{p}{[}\PY{l+m+mi}{0}\PY{p}{]}
         \PY{n}{n\PYZus{}old\PYZus{}UK} \PY{o}{=} \PY{n}{df\PYZus{}new}\PY{o}{.}\PY{n}{query}\PY{p}{(}\PY{l+s+s1}{\PYZsq{}}\PY{l+s+s1}{group == }\PY{l+s+s1}{\PYZdq{}}\PY{l+s+s1}{control}\PY{l+s+s1}{\PYZdq{}}\PY{l+s+s1}{  \PYZam{} country == }\PY{l+s+s1}{\PYZdq{}}\PY{l+s+s1}{UK}\PY{l+s+s1}{\PYZdq{}}\PY{l+s+s1}{\PYZsq{}}\PY{p}{)}\PY{o}{.}\PY{n}{count}\PY{p}{(}\PY{p}{)}\PY{p}{[}\PY{l+m+mi}{0}\PY{p}{]}
         \PY{n}{n\PYZus{}old\PYZus{}US} \PY{o}{=} \PY{n}{df\PYZus{}new}\PY{o}{.}\PY{n}{query}\PY{p}{(}\PY{l+s+s1}{\PYZsq{}}\PY{l+s+s1}{group == }\PY{l+s+s1}{\PYZdq{}}\PY{l+s+s1}{control}\PY{l+s+s1}{\PYZdq{}}\PY{l+s+s1}{  \PYZam{} country == }\PY{l+s+s1}{\PYZdq{}}\PY{l+s+s1}{US}\PY{l+s+s1}{\PYZdq{}}\PY{l+s+s1}{\PYZsq{}}\PY{p}{)}\PY{o}{.}\PY{n}{count}\PY{p}{(}\PY{p}{)}\PY{p}{[}\PY{l+m+mi}{0}\PY{p}{]}
         \PY{n}{n\PYZus{}new\PYZus{}CA} \PY{o}{=} \PY{n}{df\PYZus{}new}\PY{o}{.}\PY{n}{query}\PY{p}{(}\PY{l+s+s1}{\PYZsq{}}\PY{l+s+s1}{group == }\PY{l+s+s1}{\PYZdq{}}\PY{l+s+s1}{treatment}\PY{l+s+s1}{\PYZdq{}}\PY{l+s+s1}{  \PYZam{} country == }\PY{l+s+s1}{\PYZdq{}}\PY{l+s+s1}{CA}\PY{l+s+s1}{\PYZdq{}}\PY{l+s+s1}{\PYZsq{}}\PY{p}{)}\PY{o}{.}\PY{n}{count}\PY{p}{(}\PY{p}{)}\PY{p}{[}\PY{l+m+mi}{0}\PY{p}{]}
         \PY{n}{n\PYZus{}new\PYZus{}UK} \PY{o}{=} \PY{n}{df\PYZus{}new}\PY{o}{.}\PY{n}{query}\PY{p}{(}\PY{l+s+s1}{\PYZsq{}}\PY{l+s+s1}{group == }\PY{l+s+s1}{\PYZdq{}}\PY{l+s+s1}{treatment}\PY{l+s+s1}{\PYZdq{}}\PY{l+s+s1}{  \PYZam{} country == }\PY{l+s+s1}{\PYZdq{}}\PY{l+s+s1}{UK}\PY{l+s+s1}{\PYZdq{}}\PY{l+s+s1}{\PYZsq{}}\PY{p}{)}\PY{o}{.}\PY{n}{count}\PY{p}{(}\PY{p}{)}\PY{p}{[}\PY{l+m+mi}{0}\PY{p}{]}
         \PY{n}{n\PYZus{}new\PYZus{}US} \PY{o}{=} \PY{n}{df\PYZus{}new}\PY{o}{.}\PY{n}{query}\PY{p}{(}\PY{l+s+s1}{\PYZsq{}}\PY{l+s+s1}{group == }\PY{l+s+s1}{\PYZdq{}}\PY{l+s+s1}{treatment}\PY{l+s+s1}{\PYZdq{}}\PY{l+s+s1}{  \PYZam{} country == }\PY{l+s+s1}{\PYZdq{}}\PY{l+s+s1}{US}\PY{l+s+s1}{\PYZdq{}}\PY{l+s+s1}{\PYZsq{}}\PY{p}{)}\PY{o}{.}\PY{n}{count}\PY{p}{(}\PY{p}{)}\PY{p}{[}\PY{l+m+mi}{0}\PY{p}{]}
\end{Verbatim}


    \begin{Verbatim}[commandchars=\\\{\}]
{\color{incolor}In [{\color{incolor}81}]:} \PY{n}{z\PYZus{}score}\PY{p}{,} \PY{n}{p\PYZus{}value} \PY{o}{=} \PY{n}{sm}\PY{o}{.}\PY{n}{stats}\PY{o}{.}\PY{n}{proportions\PYZus{}ztest}\PY{p}{(}\PY{p}{[}\PY{n}{convert\PYZus{}old\PYZus{}CA}\PY{p}{,} \PY{n}{convert\PYZus{}new\PYZus{}CA}\PY{p}{]}\PY{p}{,} \PY{p}{[}\PY{n}{n\PYZus{}old\PYZus{}CA}\PY{p}{,} \PY{n}{n\PYZus{}new\PYZus{}CA}\PY{p}{]}\PY{p}{)}
         \PY{n+nb}{print}\PY{p}{(}\PY{l+s+s2}{\PYZdq{}}\PY{l+s+s2}{CA z\PYZus{}score \PYZam{} p\PYZus{}value: }\PY{l+s+s2}{\PYZdq{}}\PY{p}{,} \PY{n}{z\PYZus{}score}\PY{p}{,} \PY{n}{p\PYZus{}value}\PY{p}{)}
         \PY{n}{z\PYZus{}score}\PY{p}{,} \PY{n}{p\PYZus{}value} \PY{o}{=} \PY{n}{sm}\PY{o}{.}\PY{n}{stats}\PY{o}{.}\PY{n}{proportions\PYZus{}ztest}\PY{p}{(}\PY{p}{[}\PY{n}{convert\PYZus{}old\PYZus{}UK}\PY{p}{,} \PY{n}{convert\PYZus{}new\PYZus{}UK}\PY{p}{]}\PY{p}{,} \PY{p}{[}\PY{n}{n\PYZus{}old\PYZus{}UK}\PY{p}{,} \PY{n}{n\PYZus{}new\PYZus{}UK}\PY{p}{]}\PY{p}{)}
         \PY{n+nb}{print}\PY{p}{(}\PY{l+s+s2}{\PYZdq{}}\PY{l+s+s2}{UK z\PYZus{}score \PYZam{} p\PYZus{}value: }\PY{l+s+s2}{\PYZdq{}}\PY{p}{,} \PY{n}{z\PYZus{}score}\PY{p}{,} \PY{n}{p\PYZus{}value}\PY{p}{)}
         \PY{n}{z\PYZus{}score}\PY{p}{,} \PY{n}{p\PYZus{}value} \PY{o}{=} \PY{n}{sm}\PY{o}{.}\PY{n}{stats}\PY{o}{.}\PY{n}{proportions\PYZus{}ztest}\PY{p}{(}\PY{p}{[}\PY{n}{convert\PYZus{}old\PYZus{}US}\PY{p}{,} \PY{n}{convert\PYZus{}new\PYZus{}US}\PY{p}{]}\PY{p}{,} \PY{p}{[}\PY{n}{n\PYZus{}old\PYZus{}US}\PY{p}{,} \PY{n}{n\PYZus{}new\PYZus{}US}\PY{p}{]}\PY{p}{)}
         \PY{n+nb}{print}\PY{p}{(}\PY{l+s+s2}{\PYZdq{}}\PY{l+s+s2}{US z\PYZus{}score \PYZam{} p\PYZus{}value: }\PY{l+s+s2}{\PYZdq{}}\PY{p}{,} \PY{n}{z\PYZus{}score}\PY{p}{,} \PY{n}{p\PYZus{}value}\PY{p}{)}
\end{Verbatim}


    \begin{Verbatim}[commandchars=\\\{\}]
CA z\_score \& p\_value:  1.29689989723 0.194665631885
UK z\_score \& p\_value:  -0.474891112878 0.634864586192
US z\_score \& p\_value:  1.50519345646 0.132274347002

    \end{Verbatim}

    \begin{Verbatim}[commandchars=\\\{\}]
{\color{incolor}In [{\color{incolor}82}]:} \PY{c+c1}{\PYZsh{} Individually the location and the page they get don\PYZsq{}t show any signs of being statistically significant.}
\end{Verbatim}


    \begin{enumerate}
\def\labelenumi{\alph{enumi}.}
\setcounter{enumi}{7}
\tightlist
\item
  Though you have now looked at the individual factors of country and
  page on conversion, we would now like to look at an interaction
  between page and country to see if there significant effects on
  conversion. Create the necessary additional columns, and fit the new
  model.
\end{enumerate}

Provide the summary results, and your conclusions based on the results.

    \begin{Verbatim}[commandchars=\\\{\}]
{\color{incolor}In [{\color{incolor}84}]:} \PY{n}{log\PYZus{}mod\PYZus{}country} \PY{o}{=} \PY{n}{sm}\PY{o}{.}\PY{n}{Logit}\PY{p}{(}\PY{n}{df\PYZus{}new}\PY{p}{[}\PY{l+s+s1}{\PYZsq{}}\PY{l+s+s1}{converted}\PY{l+s+s1}{\PYZsq{}}\PY{p}{]}\PY{p}{,} \PY{n}{df\PYZus{}new}\PY{p}{[}\PY{p}{[}\PY{l+s+s1}{\PYZsq{}}\PY{l+s+s1}{intercept}\PY{l+s+s1}{\PYZsq{}}\PY{p}{,} \PY{l+s+s1}{\PYZsq{}}\PY{l+s+s1}{treatment\PYZus{}page}\PY{l+s+s1}{\PYZsq{}}\PY{p}{,} \PY{l+s+s1}{\PYZsq{}}\PY{l+s+s1}{Country\PYZus{}UK}\PY{l+s+s1}{\PYZsq{}}\PY{p}{,} \PY{l+s+s1}{\PYZsq{}}\PY{l+s+s1}{Country\PYZus{}US}\PY{l+s+s1}{\PYZsq{}}\PY{p}{]}\PY{p}{]}\PY{p}{)}
         \PY{n}{results\PYZus{}country} \PY{o}{=} \PY{n}{log\PYZus{}mod\PYZus{}country}\PY{o}{.}\PY{n}{fit}\PY{p}{(}\PY{p}{)}
         \PY{n}{results\PYZus{}country}\PY{o}{.}\PY{n}{summary}\PY{p}{(}\PY{p}{)}
\end{Verbatim}


    \begin{Verbatim}[commandchars=\\\{\}]
Optimization terminated successfully.
         Current function value: 0.366113
         Iterations 6

    \end{Verbatim}

\begin{Verbatim}[commandchars=\\\{\}]
{\color{outcolor}Out[{\color{outcolor}84}]:} <class 'statsmodels.iolib.summary.Summary'>
         """
                                    Logit Regression Results                           
         ==============================================================================
         Dep. Variable:              converted   No. Observations:               290584
         Model:                          Logit   Df Residuals:                   290580
         Method:                           MLE   Df Model:                            3
         Date:                Wed, 21 Nov 2018   Pseudo R-squ.:               2.323e-05
         Time:                        14:00:27   Log-Likelihood:            -1.0639e+05
         converged:                       True   LL-Null:                   -1.0639e+05
                                                 LLR p-value:                    0.1760
         ==================================================================================
                              coef    std err          z      P>|z|      [0.025      0.975]
         ----------------------------------------------------------------------------------
         intercept         -2.0300      0.027    -76.249      0.000      -2.082      -1.978
         treatment\_page    -0.0149      0.011     -1.307      0.191      -0.037       0.007
         Country\_UK         0.0506      0.028      1.784      0.074      -0.005       0.106
         Country\_US         0.0408      0.027      1.516      0.130      -0.012       0.093
         ==================================================================================
         """
\end{Verbatim}
            
    \begin{Verbatim}[commandchars=\\\{\}]
{\color{incolor}In [{\color{incolor}88}]:} \PY{n}{np}\PY{o}{.}\PY{n}{exp}\PY{p}{(}\PY{n}{results\PYZus{}country}\PY{o}{.}\PY{n}{params}\PY{p}{)}
\end{Verbatim}


\begin{Verbatim}[commandchars=\\\{\}]
{\color{outcolor}Out[{\color{outcolor}88}]:} intercept         0.131332
         treatment\_page    0.985168
         Country\_UK        1.051944
         Country\_US        1.041599
         dtype: float64
\end{Verbatim}
            
     \#\# Conclusions

Based on the data and odds ratio we see from the Logistic regression as
well as the data collected from the p-values and z-scores, we can see
that the location of the individual recieving the page also doesn't
influence how well the treatment page is being recieved from the control
page. With this information we can be confident that we can't not reject
the null hypothsis that the probability of the treatment page converting
an individual is the same as the control page converting an individual.


    % Add a bibliography block to the postdoc
    
    
    
    \end{document}
